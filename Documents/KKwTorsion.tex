\chapter{Kaluza--Klein reduction for gravity with torsion}

\section{Preliminaries}

%When one consider spacetimes with torsion, in order to study the dynamics of the geometry, an ansatz for the metric is not enough. In addition an ansatz for the connection is needed. However, it was shown that a general connection can be split into a Riemannian part (free of torsion), plus a contribution of the contorsion.

In what follows, in order to distinguish between the Riemannian and non-Riemannian connection. Hereon, quantities with a ring on it are Riemannian, i.e., free of torsion,
\begin{equation*}
  \spif**{a}{b} = \spif.**{a}{b} + \kf**{a}{b}.
\end{equation*}
Thus, following the computations in Sec.~\ref{sec:KKonS1}, one have that the vielbein choice,
\begin{equation*}
  \vif**{a} = 
  \begin{cases}
    \vif*{a} = e^{\alpha\phi} \vif{a}\\
    \vif*{D} = e^{\beta\phi} \left( \df[\xi] + \Ag*[1] \right)
  \end{cases}
\end{equation*}
yield the Riemannian part of the spin connection,
\begin{align}
  \spif.*{Da}{} &= \beta e^{-\alpha \phi} \pau{a} \phi \, \vif*{D} + \frac{1}{2} e^{(\beta - 2\alpha) \phi} \Fg^{a}{}_{b} \, \vif*{b}, \\
  \spif.*{ab}{} &= \spif{ab}{} + 2 \alpha e^{ - \alpha \phi } \pau{[b} \phi \, \vif*{a]} - \frac{1}{2} e^{(\beta - 2\alpha) \phi} \Fg^{a b} \, \vif*{D}.
\end{align}

However, when one deals with a theory with gravitational torsion, the Lorentz connection is not totally determined by the vielbein. Therefore, an ansatz must be proposed for the former. A general decomposition of the spin connection is of the form
\begin{equation}
  \spif*{ \hat{a} \hat{b}}{} = 
  \begin{cases}
    \spif*{a D}{} = \spif.*{a D}{} + \bml^a  + \chi^a \, \vif*{D} \\
    \spif*{a b}{} = \spif.*{ab}{}  + \bmk^{ab} + \psi^{ab} \, \vif*{D}
  \end{cases}
\end{equation}

From the spin connection ansatz and using the fact that
\begin{equation}
  \df[\vif*{D}] = \beta \, \df[\phi] \vif*{D} + e^{\beta \phi} \, \Fg*[2],
\end{equation}
one obtains
\begin{align}
  \df \spif*{a D}{}
  &= \df \spif.*{a D}{} + \df \bml^a + e^{\beta \phi} \chi^a \, \Fg*[2] + \left( \df[\chi^a] + \beta \chi^a \, \df[\phi] \right) \vif*{D}, \\
  \spif*{a}{m}  \spif*{m D}{}
  &= \spif.*{a}{m} \spif.*{m D}{} + \spif.*{a}{m} \bml^m + \bmk^a{}_m \spif.*{m D}{} + \bmk^{a}{}_m \bml^m \\
  & \quad + \left( \chi^m \, \spif.*{a}{m} + \chi^m \, \bmk^a{}_m - \psi^a{}_m  \, \spif.*{a}{D} - \psi^a{}_m  \, \bml^m \right) \vif*{D} \notag \\
  \df \spif*{a b}{}
  &= \df \spif.*{a b}{} + \df \bmk^{ab} + e^{\beta \phi} \psi^{ab} \, \Fg*[2] + \left( \df[\psi^{ab}] + \beta \psi^{ab} \, \df[\phi] \right) \vif*{D} ,\\
  \spif*{a}{m} \spif*{m}{b}
  &= \spif.*{a}{m} \spif.*{m}{b} - 2 \, \spif.*{[a}{m} \bmk^{b]m} + \bmk^a{}_m \bmk^{mb} \\
  & \quad + \left( 2 \psi^{m[b} \, \spif.*{a]}{m} + 2 \psi^{m[b} \, \bmk^{a]}{}_{m} \right) \vif*{D}, \notag \\
  \spif*{a}{D} \spif*{D}{b}
  &= \spif.*{a}{D} \spif.*{D}{b} - 2 \, \spif.*{[a}{D} \bml^{b]} - \bml^{ab} \\
  & \quad \left( 2 \chi^{[a} \, \spif.*{b]}{D} + 2 \chi^{[a} \, \bml^{b]} \right) \vif*{D}, \notag
\end{align}
from which the curvature two-form is
\begin{align}
  \rif*{a b}{} &= \rif.*{a b}{} + \free{\cdf}\bmk^{ab} + \bmk^a{}_m \bmk^{mb} - \bml^{ab} - 2 \spif.*{[a}{D} \bml^{b]} + e^{\beta \phi} \psi^{ab} \, \Fg*[2] \label{Rwtab} \\
  & \quad + \left( \free{\cdf}\psi^{ab} + \beta \psi^{ab} \, \df \phi + 2 \psi^{[a}{}_m \, \bmk^{b] m} + 2 \chi^{[a} \, \spif.*{b] D}{} + 2 \chi^{[a} \, \bml^{b]} \right) \vif*{D}  \notag \\
  \rif*{a D}{} &= \rif.*{a D}{} + \free{\cdf} \bml^a + \bmk^{a}{}_m \bml^m + \bmk^{a}{}_m \spif.*{m D}{} + e^{\beta \phi} \chi^a \, \Fg*[2] \label{Rwtad} \\
  & \quad + \left( \beta \chi^a \, \df\phi + \free{\cdf} \chi^a + \bmk^a{}_m \chi^m - \psi^a{}_m \, \spif.*{m D}{} - \psi^a{}_m \, \bml^m \right) \vif*{D}, \notag 
\end{align}

\section[Toward a Kaluza--Klein reduction on S1]{Toward a Kaluza--Klein reduction on $S^1$}

Consider the Einstein--Cartan action in $(D+1)$ dimensions,
\begin{equation}
  S_{\text{\textsc{ec}}} = \frac{1}{2 k^2} \int \frac{1}{(D-1)!} \epsilon_{ \hat{a}_0 \cdots \hat{a}_{D} } \, \rif*{\hat{a}_0 \hat{a}_1}{} \vif*{\hat{a}_2} \cdots \vif*{\hat{a}_D}.
\end{equation}
In order to find the effective lower-dimensional action, the first step is to split the indices,
\begin{equation}
  S_{\text{\textsc{ec}}} = \frac{1}{2 k^2} \int \left( \tfrac{\epsilon_{ {a}_0 \cdots {a}_{D-1} D }}{(D-2)!}  \, \rif*{{a}_0 {a}_1}{} \vif*{{a}_2} \cdots \vif*{{a}_{D-1}} \vif*{D} + \tfrac{2 \epsilon_{ D {a}_0 \cdots {a}_{D-1} }}{(D-1)!}  \, \rif*{D {a}_0}{} \vif*{{a}_1} \cdots \vif*{{a}_{D-1}} \right).
\end{equation}

Now, one notices that the structure of the curvature two-form restricts the possible contributions to the effective action. For the sake of simplicity, one could separate the curvature two-form into components lying on the $D$-dimensional spacetime ---which hereon will be called the base manifold---, and component along the extra dimension ---or fibre---,
\begin{align}
  \rif*{a b}{} &= \BM*^{ab} + \BN*^{ab} \vif*{D}, \\
  \rif*{a D}{} &= \BP*^{a} + \BQ*^{a} \vif*{D}.
\end{align}
Notice that from Eqs.~\eqref{Rwtab} and~\eqref{Rwtad} explicit form for $\BM*^{ab}$, $\BN*^{ab}$, $\BP*^a$ and $\BQ*^a$. Nonetheless, a subtle point is to remember that in this decomposition both $\rif.*{a b}{}$ and $\rif.*{a D}{}$, posses components on the base and the fibre.

From the above decomposition, it follows that
\begin{align}
  S_{\text{\textsc{ec}}} = \frac{1}{2 k^2} \int \left( \tfrac{\epsilon_{ {a}_0 \cdots {a}_{D-1} D }}{(D-2)!}  \, \BM*^{ a_0 a_1} \vif*{a_2} \cdots \vif*{a_{D-1}} \vif*{D} + \tfrac{2 \epsilon_{ {a}_0 \cdots {a}_{D-1} D }}{(D-1)!}  \, \BQ*^{a_0} \vif*{{a}_1} \cdots \vif*{{a}_{D-1}} \vif*{D} \right)
\end{align}


%%%%%%%%%%%%%%%%%%%%%%%%%%%%%%%
\nocite{German:1993bq,Aros:2007nn}
%% Lovelock--Cartan
\nocite{MuellerHoissen:1989yv,Mardones:1990qc}
