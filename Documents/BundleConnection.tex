%%%%%%%%%%%%%%%%%% BUNDLES AND CONNECTION

\chapter{Bundles and Connections}

So far, tangent and cotangent bundles have been considered, and other generalisations have been slightly appearing.

Instead of attacking the problem formally, in the following, a heuristic introduction to more general bundles is shown, and also the definition of connections on these bundles.

\section{Fibre Bundles}

A \emph{\idx{Bundle}} is a geometric structure composed by two manifolds $E$ (the bundle) and $M$ (the base)\footnote{In some sense the base manifold is a ``piece'' of the bundle, i.e., $M \subset E$.} together with a projection map $\pi: E \to M$. If $U \subset M$, then locally, $E \simeq U \times F$, where $F$ is a manifold called the fibre of $E$. 

\subsubsection*{Examples}
\begin{itemize}
\item A cylinder is a bundle $E$ with base manifold $M = \R$ and fibre $F = S^1$, s.t. $\pi: \R \times S^1 \to \R$.
  
  In this case, $E = M \times F$ globally, then the bundle is called \emph{Trivial}\index{Bundle!trivial}.
\item If one takes an interval $I \subset \R$, the ``finite'' cylinder is a trivial bundle $E = I \times S^1$ or $E = S^1 \times I$.
\item The simplest non-trivial bundle is a M\"obius strip, which is a bundle $E$ locally homeomorphic to $S^1 \times I$, but is not globally $S^1 \times I$ because there is a ``twist'' (a bit more formally there is a $Z_2$-action on the fibre).
  \begin{center}
    \includegraphics{Pict/tikz-cylinder.pdf}
    \hspace{5em}
    \includegraphics{Pict/tikz-moebius.pdf}
  \end{center}
\item  A 2-torus $T^2$ is a trivial bundle with base manifold $M = S^1$ and fibre $F = S^1$.
\end{itemize}

In essence there are three types of bundles:
\begin{description}
\item [Fibre Bundle]\index{Bundle!fibre} is a general bundle structure, i.e., a manifold which locally looks like a product of a base manifold and the fibre.
\item [Vector Bundle]\index{Bundle!vector} A fibre bundle whose fibre is a vector space.
\item [Principal Bundle]\index{Bundle!principal} A fibre bundle whose fibre is a manifold with group structure, i.e., the fibre is a Lie group manifold.
\end{description}

\subsection{Sections on Bundles}

A \idx{Section}, $\psi$, on a bundle $E \xrightarrow{\pi} M$ is a smooth map from the base manifold to the bundle,
\begin{align}
  \psi: M \to E,  
\end{align}
such that $\pi \circ \psi = \id_M$.

Sections are one important object for physicist, because they represent the physical fields of a theory.

\subsection{More About Principal Bundles}

As stated before, principal bundles are fibre bundles whose fibres are a Lie Group. Therefore, there is an action of $G$ on $F$. They are the natural framework for the study of Gauge Theories.

\subsubsection*{Associated Bundles}

The associated bundles are constructed from  principal bundles. Let $P(M,G)$ be a principal bundle, with fibre $F$ admitting and action of the Lie group $G$.

Given a pair $(u,f)$ on $P \times F$, a bundle $\(P \times F\)/G$ can be construct by identifying points related by the action of the group $G$, 
\begin{align}
  (u,f)\sim \( u g, g^{-1} f \).
\end{align}
This construction is known as \emph{associated bundle}\index{Bundle!associated}.

More generally, the fibre can admit the action of a representation $\rho$ of $G$. In this case, the associated bundle $\(P \times_\rho F\)/G$ is constructed through,
\begin{align}
  (u,f) \sim \( u g, \rho\(g^{-1}\) f \).
\end{align}


\section{Connections on the Tangent Bundle}

A connection on the tangent bundle is an application $\nabla: TM \times TM \to TM$, defined by
\begin{align}
  \nabla: (X,Y) \to \nabla_X Y,
\end{align}
satisfying 
\begin{align}
  \nabla_{f X} Y &=f \nabla_X Y\\
  \nabla_X f Y &= f\nabla_X Y + Y \cdot X[f],\\
  \nabla_{X + Y} Z &= \nabla_X Z + \nabla_Y Z.
\end{align}

On a vector basis, 
\begin{align}
  \nabla_{\partial_i} \partial_j \equiv \nabla_i \partial_j = \conn{i}{k}{j} \pa{k},
\end{align}
then, in components
\begin{align}
  \nab{i} Y=\(\pa{i} Y^j + \conn{i}{j}{k} Y^k\) \pa{j}.
\end{align}
In physics, the term between the brackets is called the covariant derivative of a vector.


\begin{infobox}[frametitle={General Connections}]
  In a general bundle $E$, the connections are applications
  \begin{align*}
    \nabla: TM \otimes \mathcal{E} \to \mathcal{E},
  \end{align*}
  where $\mathcal{E}$ is the space of sections on the bundle $E$.
\end{infobox}

\section{Parallel Transport and Geodesic}

Let $V$ be a tangent vector to a curve, i.e., 
\begin{align}
  V = \der{x^\mu}{ t} \(c(t)\) \;\pa{\mu}\bigr|_{c(t)},
\end{align}
then $X\in TM$ is said to be parallel transported\index{Parallel transport} along $c(t)$ if
\begin{align}
  \nabla_V X =0\quad \forall\;t\in I.
\end{align}

If the tangent vector to a curve ($V$) satisfies the parallel transport condition,
\begin{align}
  \nabla_V V =0,
\end{align}
then the curve $c(t)$ is called a \emph{\idx{Geodesic}}.

\begin{Ebox}
  \begin{itemize}
  \item Use that
    \begin{align*}
      \nab{i} \pa{j} &= \conn{i}{k}{j} \pa{k}\\
      \nab{i}' \pa{j}' &= \(\Ga_{i}'\)^{k}{}_{j} \pa{k}',
    \end{align*}
    to find the transformation rule of $\Ga$'s.
  \item Find the action of $\nabla$ on 1-forms and rank two tensors.
  \item Write in coordinates the condition of parallel transport and geodesic curve.
  \end{itemize}
\end{Ebox}


\section{Torsion and Curvature}

\begin{align}
  T(X,Y) &= \nab{X}Y-\nab{Y}X-\comm{X}{Y}\\
  R(X,Y,Z) &= \nab{X}\nab{Y}Z -\nab{Y}\nab{X}Z -\nab{\comm{X}{Y}}Z.
\end{align}

\begin{Ebox}
  \begin{itemize}
  \item Find coordinate expressions for $T(X,Y)$ and $R(X,Y,Z)$.
  \item Show that they are multilinear objects, i.e., they are tensors.
  \end{itemize}
\end{Ebox}
\bigskip
\begin{infobox}[frametitle={NOTE}]
  In general the concept of curvature can be extended to sections on a bundle, where 
  \begin{align*}
    \nabla: TM \otimes \mathcal{E} \to \mathcal{E},
  \end{align*}
  while the torsion cannot, since $X$ and $Y$ are necessarily the same kind of object.
\end{infobox}


