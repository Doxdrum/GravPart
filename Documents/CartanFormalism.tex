%%%%%%%%%%%%%%%%%% ANOTHER LECTURE

\chapter{Cartan Calculations in (semi)Riemannian Geometry}

Before continuing, a change of notation will be made. Although this can be confusing, the previous notation is used often by mathematicians, while the following is used by physicist.

\section{Frame Fields in (semi)Riemannian Geometry}

%Before continuing, a change of notation will be made. Although this can be confusing, the previous notation is used often by mathematicians, while the following is used by physicist.

Let $M$ be a manifold, and $g$ a (semi)Riemannian metric defined on $M$. Then the line element for the metric $g$ is
\begin{align}
  \de{s}^2(g) = g_{\mu\nu} \de{x}^\mu \otimes \de{x}^\nu.
\end{align}
Nonetheless, the information about the metric structure of the manifold can be translate to the languages of frames,
\begin{equation}
  \begin{split}
    \de{s}^2(g)
    &= g_{\mu\nu} \de{x}^\mu \otimes \de{x}^\nu \\
    &= \eta_{ab} \; \vi{a}{\mu}(x) \, \vi{b}{\nu}(x) \; \de{x}^\mu \otimes \de{x}^\nu \\
    &= \eta_{ab} \; \vif{a} \otimes \vif{b}.
  \end{split}
\end{equation}
Therefore, the vielbein, $\vif{a}$, is the fields equivalent to the metric tensor. In order to complete the structure, one needs information about the transport of geometrical objects lying on bundles based on $M$. That information is encoded on the spin connection, $\spif{a}{b}$. Using these quantities one finds  the generalisation of the structure equations of Cartan,
\begin{align}
  \df \vif{a} + \spif{a}{b} \we \vif{b} &= \Tf{a}, \label{firstSE}\\
  \df \spif{a}{c} + \spif{a}{b} \we \spif{b}{c} &= \Rif{a}{c} \label{secondSE}.
\end{align}
In the sense of the previous lesson, the torsion 2-form, $\Tf{a}$, and the curvature 2-form,$\Rif{a}{c}$, measure the impossibility of endow an Euclidean structure on $M$.

\subsection*{Remarks on the Structure Equations}

\begin{itemize}
\item In general relativity, the torsion vanishes. Therefore, the first Cartan's structure Eq.~\eqref{firstSE}, yields an explicit expression for the spin connection as a function of the vielbein. This is equivalent to the well-known expression of the Levi-Civita connection $\Gamma_\mu{}^\lambda{}_\nu$ as a function of the metric and its derivatives.
\item The components of the torsion and curvature 2-forms are given by
%%%%
%% Change the eq. of the torsion (to be the equiv of the curvature
%%%%
  \begin{align}
    \Tf{a} &= \frac{1}{2} \tors{\mu}{a}{\nu} \; \de{x}^\mu \we \de{x}^\nu,
    \label{def-tor-comp} \\
    %\tor_\mu{}^\lambda{}_\nu \;\df\; x^\mu\w\df\; x^\nu &= \tor_a{}^c{}_b \vin{\lambda}{c}\;\vif{a}\w\vif{b},\\
    \Rif{a}{b} &= \frac{1}{2} \Ri^a{}_{b\mu\nu} \; \de{x}^\mu \we \de{x}^\nu.
    \label{def-curv-comp}
  \end{align}
\end{itemize}

\section{Playing with the Structure Equations}

From the first S.E.
\begin{align*}
  \df \( \df \vif{a} + \spif{a}{b} \we \vif{b} \) &= \df \Tf{a} \\
  \therefore \quad
  \df \Tf{a} &= \( \df \spif{a}{b}\) \we \vif{b} - \spif{a}{b} \we \df \vif{b}  \\
  \df \Tf{a} &= \( \Rif{a}{b} - \spif{a}{c} \we \spif{c}{b}\) \we \vif{b} - \spif{a}{c} \we \df \vif{c} \\
  \df \Tf{a} &= \Rif{a}{b} \we \vif{b} - \spif{a}{c} \we \( \spif{c}{b} \we \vif{b} + \df \vif{c} \) \\
  \df \Tf{a} &= \Rif{a}{b} \we \vif{b}-\spif{a}{c} \we \Tf{c},
  %  \Rif{a}{b} \we \vif{b}&=\cdf\Tf{a}
\end{align*}
then,
\begin{align}
  \Rif{a}{b} \we \vif{b}&=\df\Tf{a}+\spif{a}{c} \we \Tf{c}.
\end{align}


From the second S.E.
\begin{align*}
  \df\(\df\spif{a}{b}+\spif{a}{c} \we \spif{c}{b}\) &= \df\Rif{a}{b}\\
  \therefore \quad
  \df \Rif{a}{b} &= \(\df\spif{a}{c}\) \we \spif{c}{b}-\spif{a}{c} \we \(\df\spif{c}{b}\) \\
  \df \Rif{a}{b} &= \Rif{a}{c} \we \spif{c}{b}-\spif{a}{c} \we \Rif{c}{b},
\end{align*}
i.e.,
\begin{align}
  \df\Rif{a}{b} + \spif{a}{c} \we \Rif{c}{b} - \Rif{a}{c} \we \spif{c}{b}   &= 0.
\end{align}


\begin{WEbox}[frametitle={Curvature of the 2-sphere},
  frametitlerule=true,
  frametitlealignment=\centering,
  frametitleaboveskip=10pt,]
  A 2-sphere  of radius $R$  has a line element given by
  \begin{align}
    \de{s}^2(g) = R^2 \( \de{\theta} \otimes \de{\theta} + \sin^2(\theta) \de{\phi} \otimes \de{\phi}\),
  \end{align}
  which can be written in terms of vielbeins as
  \begin{align}
    \de{s}^2(g) = \delta_{ij} \; \vif{i} \otimes \vif{j},
  \end{align}
  with $$\vif{1} = R \de{\theta}$  and $\vif{2} = R \sin(\theta) \de{\phi}.$$

  Using the first structure equation one gets,
  \begin{align}
    \df \vif{2}
    &=  R \cos(\theta) \de{\theta} \we \de{\phi} \notag \\
    &= -R \cos(\theta) \de{\phi} \we \de{\theta} \notag \\
    &= - \cos(\theta)  \de{\phi} \we \vif{1},
  \end{align}
  then
  \begin{align}
    \spif{2}{1} = \cos(\theta) \de{\phi}.
  \end{align}

  From the second structure equation, one obtains the only non-vanishing component of the 2-form curvature
  \begin{align}
    \Rif{2}{1} = -\Rif{1}{2} = \df \spif{2}{1} = - \sin(\theta) \de{\theta} \we \de{\phi}.
  \end{align}
  It follows from the last equation that
  \begin{align}
    \Rif{12}{} = \frac{1}{R^2}\vif{1} \we \vif{2},\label{curv-2sphere}
  \end{align}
  then, the 2-sphere is a manifold with constant curvature, as in Eq.~\eqref{def-const-curv}, with (Gau\ss{}ian) curvature $\kappa=\frac{1}{R^2}$.

  From Eqs.~\eqref{def-curv-comp} and \eqref{curv-2sphere}, one obtains the non-vanishing component of the Riemann tensor,
  \begin{align}
    \Ri_{\theta\phi}{}^1{}_2 = \sin(\theta)\quad\Rightarrow\quad \Ri_{\theta\phi}{}^\theta{}_\phi = \sin^2(\theta).
  \end{align}
  Now, using the metric to raising and lowering indeces, and contracting, the Ricci and Ricci scalar are obtained,
  \begin{align}
    \Ri_{\theta\theta}&= 1, & \Ri_{\phi\phi}&= \sin^2(\theta),\notag\\
    \Ri&= \frac{2}{R^2}. & &
  \end{align}

  Finally, note that the 2-sphere is an Einstein manifold, since
  \begin{align}
    \Ri_{\mu\nu} = \frac{1}{R^2}g_{\mu\nu}.
  \end{align}
\end{WEbox}

\begin{WEbox}[frametitle={Curvature of the 2-hyperboloid},
  frametitlerule=true,
  frametitlealignment=\centering,
  frametitleaboveskip=10pt,]
  The line element of a 2-hyperboloid is very similar to the one on the 2-sphere,
  \begin{align}
    \de{s}^2(g) = R^2\(\de{\theta}\otimes\de{\theta}+\sinh^2(\theta)\de{\phi}\otimes\de{\phi}\),
  \end{align}
  then, the calculation of the curvature can be followed straightforward from the previous example. Thus, only the results are show,
  \begin{align}
    \vif{1}&= R\de{\theta} & \vif{2} &= R\sinh(\theta)\de{\phi}\\
    \spif{2}{1}&= \cosh(\theta)\de{\phi} & \Rif{2}{1} &= \sinh(\theta)\de{\theta} \we \de{\phi}\\
    \Ri_{\phi\theta}{}^2{}_1 &= -\sinh(\theta) & \Ri_{\phi\theta}{}^\phi{}_\theta&= -1\\
    \Ri_{\theta\theta}&= -1 & \Ri_{\phi\phi}&= -\sinh^2(\theta).
  \end{align}
  The Ricci scalar is 
  \begin{align*}
    \Ri = -\frac{2}{R^2},
  \end{align*}
  and the hyperboloid is a manifold of constant curvature,
  \begin{align*}
    \Rif{12}{} = -\frac{1}{R^2}\vif{1} \we \vif{2},
  \end{align*}
  i.e., the Gau\ss{}ian curvature of the hyperboloid is $\kappa=-\frac{1}{R^2}$.
\end{WEbox}


\begin{WEbox}
  The curvature of the Anti de Sitter ($AdS$) spacetime is calculated, using Cartan's formalism.
  
  The metric of the $AdS$ spacetime is 
  \begin{align}
    \de{s}^2(g) = \frac{1}{z^2}\(-\df\; t\otimes\df\; t +\df\; x\otimes\df\; x+\df\; y\otimes\df\; y+\df\; z\otimes\df\; z\),
  \end{align}
  then the vielbeins are
  \begin{align}
    \vif{0} &= \frac{\df\; t}{z}, & \vif{1} &= \frac{\df\; x}{z},\notag\\
    \vif{2} &= \frac{\df\; y}{z}, & \vif{3} &= \frac{\df\; z}{z},
  \end{align}
  then,
  \begin{align}
    \df\vif{0}&=-\frac{1}{z^2}\df\; z \we \df\; t & \df\vif{1}&=-\frac{1}{z^2}\df\; z \we \df\; x\notag\\
    &= \frac{1}{z}\df\; t \we \vif{3} & &= \frac{1}{z}\df\; x \we \vif{3}\\
    \df\vif{2}&=-\frac{1}{z^2}\df\; z \we \df\; y & \df\vif{3}&=0\notag \\
    &= \frac{1}{z}\df\; y \we \vif{3} & & \notag
  \end{align}
  therefore,
  \begin{align}
    \spif{0}{3} = -\frac{1}{z}\df\; t,\quad \spif{1}{3}=-\frac{1}{z}\df\; x,\quad \spif{2}{3} = -\frac{1}{z}\df\; y.
  \end{align}
  Now, using the second structure equation, Eq. (\ref{secondSE}), it follows that
  \begin{align}
    \Rif{0}{3} &= \df\spif{0}{3} = \frac{1}{z^2}\df\; z \we \df\; t &  \Rif{1}{3} &= \df\spif{1}{3} = \frac{1}{z^2}\df\; z \we \df\; x \notag\\
    \Rif{2}{3} &= \df\spif{2}{3} = \frac{1}{z^2}\df\; z \we \df\; y &  \Rif{0}{1} &= \spif{0}{3} \we \spif{3}{1} = -\frac{1}{z^2}\df\; t \we \df\; x \\
    \Rif{0}{2} &= \spif{0}{3} \we \spif{3}{2} = -\frac{1}{z^2}\df\; t \we \df\; y & \Rif{1}{2} &= \spif{1}{3} \we \spif{3}{2} = -\frac{1}{z^2}\df\; x \we \df\; y.\notag
  \end{align}
  
  Now, the non-vanishing components of the Riemann tensor are,
  \begin{align}
    \Ri^t{}_{ztz} &= -\frac{1}{z^2} & \Ri^x{}_{zxz} &= -\frac{1}{z^2}\notag\\
    \Ri^y{}_{zyz} &= -\frac{1}{z^2} & \Ri^t{}_{xtx} &= -\frac{1}{z^2}\\
    \Ri^t{}_{yty} &= -\frac{1}{z^2} & \Ri^x{}_{yxy} &= -\frac{1}{z^2}\notag\\
  \end{align}
  and finally, the Ricci tensor is
  \begin{align}
    \Ri_{\mu\nu} = -3 g_{\mu\nu}.
  \end{align}

  The $AdS$ spacetime is an example of an Einstein space, or in other words it is a solution of Einstein's equations with cosmological constant, $\Lambda=-3$.
\end{WEbox}

\begin{Ebox}
  Consider a Riemannian manifold with a metric whose line element is
  \begin{align}
    \de{s}^2(g) = -e^{2\Xi(r)}\df\; t\otimes \df\; t+ e^{-2\Xi(r)}\df\; r\otimes\df\; r+ r^2\(\de{\theta}\otimes\de{\theta}+\sin^2(\theta)\de{\phi}\otimes\de{\phi}\),
  \end{align}
  Find the function $\Xi(r)$ which solves the Einstein's equation in the vacuum.
\end{Ebox}



\section[Hilbert-Einstein Action]{Hilbert-Einstein Action (\`a la Cartan)}



\begin{Pro}
  \begin{align*}
    S_{gr} = \frac{1}{2\kappa^2}\int\frac{\epsilon_{a_1\cdots a_D}}{(D-2)!}\Rif{a_1 a_2}{} \we \vif{a_3} \we \cdots \we \vif{a_D}=\frac{1}{2\kappa^2}\int\dn{D}{x}\abs{e}\Ri.
  \end{align*}
\end{Pro}


\begin{proof}
 % \begin{small}
    \begin{align*}
      \frac{\epsilon_{a_1\cdots a_D}}{(D-2)!}\Rif{a_1 a_2}{} \we \vif{a_3} \we \cdots \we \vif{a_D} &= \frac{\epsilon_{a_1\cdots a_D}}{(D-2)!}\frac{1}{2}\Ri^{a_1 a_2}{}_{cd}\;\vif{c} \we \vif{d} \we \vif{a_3} \we \cdots \we \vif{a_D}\\
      &= -\frac{\epsilon_{a_1\cdots a_D}}{(D-2)!}\frac{1}{2}\Ri^{a_1 a_2}{}_{cd}\;\epsilon^{c d a_3\cdots a_D}dV\\
      &= \frac{1}{2}\Ri^{a_1 a_2}{}_{cd}\(\delta_{a_1}^c\delta_{a_2}^d -\delta_{a_1}^d\delta_{a_2}^c \)\,dV\\
      &= \dn{D}{x}\abs{e}\Ri
    \end{align*}
  %\end{small}
\end{proof}

\section{Example of a non-Trivial Calculation}

Consider the metric whose line-element is
\begin{align*}
  \de{s}^2(g) = A^2\[-\frac{(p^2\df\; \tau +\df\; \sigma)^2}{B^2}+\frac{(\df\; \tau - q^2\df\; \sigma)^2}{C^2}+B^2\df\; q^2 +C^2\df\; p^2\],
\end{align*}
with $A,B,C$ function on $p$ and $q$.

Define,
\begin{align*}
  \bs{\tht}^0 &= p^2\df\; \tau +\df\; \sigma,\\
  \bs{\tht}^1 &= \df\; \tau - q^2\df\; \sigma.
\end{align*}
Then,
\begin{align*}
  \df\; \tau &= \frac{q^2\bs{\tht}^0 +\bs{\tht}^1}{1+p^2q^2},\\
  \df\; \sigma &= \frac{\bs{\tht}^0 -p^2\bs{\tht}^1}{1+p^2q^2}.
\end{align*}
Additionally,
\begin{align}
  \df\; \bs{\tht}^0 &= \df\(p^2\df\; \tau +\df\; \sigma\)\notag\\
  &= 2p \,\df\; p \we \df\; \tau\notag\\
  &= \frac{2p}{1+p^2q^2}\(q^2\, \df\; p \we \bs{\tht}^0 +\df\; p \we \bs{\tht}^1\)\\
  \df\; \bs{\tht}^1 &= \df\(\df\; \tau -q^2\df\; \sigma\)\notag\\
  &= -2q \,\df\; q \we \df\; \sigma\notag\\
  &=-\frac{ 2q}{1+p^2q^2}\(\df\; q \we \bs{\tht}^0 - p^2\,\df\; q \we \bs{\tht}^1\).
\end{align}


Choosing the vielbeins by,
\begin{align}
  \vif{0} &= \frac{A}{B}\,\bs{\tht}^0,\\
  \vif{1} &= \frac{A}{C}\,\bs{\tht}^1,\\
  \vif{2} &= AB\,\df\; q,\\
  \vif{3} &= AC\,\df\; p,
\end{align}
their exterior derivative are
\begin{align}
  \df\vif{0} &= \partial_p\(\frac{A}{B}\)\, \df\; p \we \bs{\tht}^0 + \partial_q\(\frac{A}{B}\)\, \df\; q \we \bs{\tht}^0 + \frac{A}{B}\,\df\; \bs{\tht}^0,\\
  \df\vif{1} &= \partial_p\(\frac{A}{C}\)\, \df\; p \we \bs{\tht}^1 + \partial_q\(\frac{A}{C}\)\, \df\; q \we \bs{\tht}^1 + \frac{A}{C}\,\df\; \bs{\tht}^1,\\
  \df\vif{2} &= \partial_p(AB) \,\df\; p \we \df\; q ,\\
  \df\vif{3} &= \partial_q(AC)\, \df\; q \we \df\; p.
\end{align}

Now, 
\begin{align}
  \df\vif{2} &= \partial_p(AB) \,\df\; p \we \df\; q \notag\\
  &= -\partial_p(AB) \,\df\; q \we \df\; p \notag\\
  &= -\frac{\partial_p(AB)}{AC} \,\df\; q \we \vif{3} \label{DLspi23}\\
  &= -\spif{2}{a} \we \vif{a},\notag
\end{align}
and
\begin{align}
  \df\vif{3} &= \partial_q(AC) \,\df\; q \we \df\; p \notag\\
  &= -\partial_q(AC) \,\df\; p \we \df\; q \notag\\
  &= -\frac{\partial_q(AC)}{AB} \,\df\; p \we \vif{2} \label{DLspi32}\\
  &= -\spif{3}{a} \we \vif{a}.\notag
\end{align}
Note that one needs to combine Eqs. \eqref{DLspi23} and \eqref{DLspi32}, to obtain a consistent spin connection $\spif{23}{}=-\spif{32}{}$. The final result is therefore,
\begin{align}
  \spif{2}{3}= \spif{23}{} = -\spif{32}{}=-\spif{3}{2}&= \frac{\partial_p(AB)}{AC} \,\df\; q - \frac{\partial_q(AC)}{AB} \,\df\; p \\
  &= \frac{1}{A^2BC} \[\partial_p(AB)\,\vif{2} - {\partial_q(AC)}\,\vif{3}\].\notag
\end{align}

From
\begin{align}
  \df\vif{0} &= \partial_p\(\frac{A}{B}\)\, \df\; p \we \bs{\tht}^0 + \partial_q\(\frac{A}{B}\)\, \df\; q \we \bs{\tht}^0 + \frac{A}{B}\,\df\; \bs{\tht}^0\notag\\
  &= -\partial_p\(\frac{A}{B}\)\, \bs{\tht}^0 \we \df\; p - \partial_q\(\frac{A}{B}\)\, \bs{\tht}^0 \we \df\; q + \frac{2p}{1+p^2q^2}\frac{A}{B}\(q^2\, \df\; p \we \bs{\tht}^0 +\df\; p \we \bs{\tht}^1\)\notag\\
  %% &= -\frac{1}{AC}\partial_p\(\frac{A}{B}\)\, \bs{\tht}^0 \we \vif{3} - \frac{1}{AB}\partial_q\(\frac{A}{B}\)\, \bs{\tht}^0 \we \vif{2} + 2p\frac{A}{B}\,\frac{q^2\, \df\; p \we \bs{\tht}^0 +\df\; p \we \bs{\tht}^1}{1+p^2q^2}\notag\\
  %% &= -\frac{1}{A^2BC}\partial_p\(\frac{A}{B}\)\, \vif{0} \we \vif{3} - \frac{1}{A^2}\partial_q\(\frac{A}{B}\)\, \vif{0} \we \vif{2}\notag \\
  %% &{\phantom{=\;}} - 2p q^2\frac{A}{B}\,\frac{\bs{\tht}^0 \we \df\; p }{1+p^2q^2}+ 2p\frac{A}{B}\,\frac{\df\; p \we \bs{\tht}^1}{1+p^2q^2}\notag\\
  &= -\[\frac{B}{A^2C}\partial_p\(\frac{A}{B}\) + \frac{1}{AC}\,\frac{2p q^2}{1+p^2q^2}  \]\, \vif{0} \we \vif{3} - \frac{1}{A^2}\partial_q\(\frac{A}{B}\)\, \vif{0} \we \vif{2}\notag \\
  &{\phantom{=\;}} + \frac{2p}{1+p^2q^2}\frac{1}{AB}\;\vif{3} \we \vif{1}\\
  &= -\spif{0}{a} \we \vif{a},\notag
\end{align}
and,
\begin{align}
  \df\vif{1} &= \partial_p\(\frac{A}{C}\)\, \df\; p \we \bs{\tht}^1 + \partial_q\(\frac{A}{C}\)\, \df\; q \we \bs{\tht}^1 + \frac{A}{C}\,\df\; \bs{\tht}^1\notag\\
  &= \partial_p\(\frac{A}{C}\)\, \df\; p \we \bs{\tht}^1 + \partial_q\(\frac{A}{C}\)\, \df\; q \we \bs{\tht}^1 - \frac{2q}{1+p^2q^2}\frac{A}{C}\, \(\df\; q \we \bs{\tht}^0 - p^2\,\df\; q \we \bs{\tht}^1\)\notag\\
  &= -\frac{1}{A^2}\partial_p\(\frac{A}{C}\)\, \vif{1} \we \vif{3} - \[\frac{C}{A^2B}\partial_q\(\frac{A}{C}\) + \frac{2q p^2}{1+p^2q^2}\frac{1}{AB}\]\,\vif{1} \we \vif{2} \notag\\
  &{\phantom{=\;}} - \frac{2q}{1+p^2q^2}\frac{1}{AC}\, \vif{2} \we \vif{0} ,\\
  &= -\spif{1}{a} \we \vif{a},\notag
\end{align}
one obtains
\begin{align}
  \spif{0}{1} &= - \frac{p}{1+p^2q^2}\frac{1}{AB}\,\vif{3} + f_{01}\\
  \spif{0}{2} &= \frac{1}{A^2}\partial_q\(\frac{A}{B}\)\, \vif{0} - f_{02}\\
  \spif{0}{3} &=  \[\frac{B}{A^2C}\partial_p\(\frac{A}{B}\) + \frac{2p q^2}{1+p^2q^2}\frac{1}{AC}\]\, \vif{0} + \frac{p}{1+p^2q^2}\frac{1}{AB}\,\vif{1} - f_{03},\\
  \spif{1}{0} &= \frac{q}{1+p^2q^2}\frac{1}{AC}\, \vif{2} -f_{10},\\
  \spif{1}{2} &=\[\frac{C}{A^2B}\partial_q\(\frac{A}{C}\)+\frac{2qp^2}{1+p^2q^2}\frac{1}{AB}\]\,\vif{1} -\frac{q}{1+p^2q^2}\frac{1}{AC}\, \vif{0} - f_{12},\\
  \spif{1}{3} &= \frac{1}{A^2}\partial_p\(\frac{A}{C}\)\, \vif{1} - f_{13},
\end{align}
where the 1-forms $f_{ab}$ will be adjusted to get a consistent set of spin connection 1-forms.
