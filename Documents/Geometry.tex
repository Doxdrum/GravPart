%% \chapter{Differential Geometry}

%% \section{Superalgebras}

%% A ({\it commutative associative}) superalgebra, $A=A_{0}\oplus A_{1}$,  is a vector space over $\mathbb{K}$, endowed with a bilinear map $A\otimes A\to A$ satisfying, 
%% \begin{align}
%%   A_{0}\oplus A_{0}&\to A_{0}\\
%%   A_{0}\oplus A_{1}&\to A_{1}\\
%%   A_{1}\oplus A_{1}&\to A_{0}
%% \end{align}

%%%%%%%%%%%%%%%%%% CALCULUS ON MANIFOLDS

\chapter{Calculus on Manifolds}

\section{What is a Manifold?}

A $m$-dimensional {\sc Manifold} is a (topological) space which at each point looks locally like $\R^m$. Note that $m$ is a fixed number, and defines the dimension of the manifold.

\begin{Def}[Manifold]
  A {\sc Manifold}, $M$, is a topological space homeomorphic to $\R^m$.
\end{Def}

Examples of manifolds are  the body of a cylinder,  the spheres ($S^m$), the torus ($T^m$).

More illustrative is an example of a non-manifold, such as the body of a cone. A cone is not a manifold because the neighbourhood of the apex looks not line $\R^2$.

\section{Geometrical Objects on Manifolds}

\subsection{Vectors}

\subsection{One-Forms}

\subsection{Tensors}

\begin{Ebox}
  How do the components of $T$ transform? Where 
  \begin{align}
    T= T^{a_1\cdots a_p}{}_{b_1\cdots b_q} \partial_{a_1}\otimes\partial_{a_p}\otimes dx^{b_1}\otimes dx^{b_q}.
  \end{align}
\end{Ebox}

\section{Induced Maps and Submanifolds}

\subsection{Differential Map}

\subsection{Pullback Map}

\begin{Ebox}
  Let $M$, $N$ and $P$ be manifolds, and $ f:M\to N$, $g:N\to P$,  be maps between manifolds. 

  Show
  \begin{align*}
    (g\circ f)_* &= g_*\circ f_*\\
    (g\circ f)^* &= g^*\circ f^*
  \end{align*}
\end{Ebox}

\subsection{Submanifolds}

\section{Flows and Lie Derivative}

\begin{Ebox}
  For the example with $X=x\partial_y -y\partial_x$, show explicitly that 
  \begin{align}
    \sigma_t\circ \sigma_s = \sigma_{t+s}.
  \end{align}
\end{Ebox}


\begin{Ebox}
  Find the flow generated by
  \begin{align}
    X = x\partial_y +y\partial_x.
  \end{align}
\end{Ebox}

\begin{Ebox}
  \begin{itemize}
  \item Let $X$ and $Y$ be vector fields on $M$ and $f:M\to\R$ a function.
    \begin{itemize}
    \item Calculate $\Li_{X}f$.
    \item Calculate $\Li_{fX}Y$.
    \item Calculate $\Li_{X}fY$.
    \end{itemize}
  \item  Let $X$ and $Y$ be vector fields on $M$ and $f:M\to N$ a map between manifolds. Show that
    \begin{align}
      f_*\comm{X}{Y} =\comm{f_*X}{f_*Y}.
    \end{align}
  \item Let $\omega\in\Lambda^1(M)$ be a one-form on $M$. Calculate
    \begin{align}
      \Li_X \omega
    \end{align}
  \end{itemize}
\end{Ebox}




%%%%%%%%%%%%%%%%%% DIFFERENTIAL FORMS

\chapter{Differential Forms}

\begin{WEbox}[frametitle={Differential Forms in $\R^3$},
  frametitlerule=true,
  frametitlealignment=\centering,
  frametitleaboveskip=10pt,]
  In three dimensions the maximum order of a differential form is 3. Therefore, the complete set of forms on $\R^3$ is given by
  \begin{align*}
    \Lambda^0(\R^3) &=\F(\R^3), & \Lambda^1(\R^3) &= T^*\R^3\simeq \R^3,\notag\\
    \Lambda^2(\R^3) &= T^*\R^3\w T^*\R^3\simeq \R^3, & \Lambda^3(\R^3) &=T^*\R^3\w T^*\R^3\w T^*\R^3\simeq\F(\R^3).
  \end{align*}
  
  Let $f\in\Lambda^0(\R^3) $, then the exterior derivative of $f$ yields
  \begin{align}
    \df f= \frac{\partial f}{\partial x}\df x +\frac{\partial f}{\partial y}\df y +\frac{\partial f}{\partial z}\df z,
  \end{align}
  which is the equivalent of the gradient operation.
  
  Let $\omega\in \Lambda^1(\R^3)$ with components $\omega = \omega_x\df x+\omega_y\df y+\omega_z\df z$, then its exterior derivative yields,
  \begin{align}
    \df\omega &= \(\partial_x\omega_y-\partial_y\omega_x\)\df x\w\df y +\(\partial_y\omega_z-\partial_z\omega_y\)\df y\w\df z\notag\\
    &\phantom{==}+\(\partial_z\omega_x-\partial_x\omega_z\)\df z\w\df x, 
  \end{align}
  which is the equivalent of the curl of $\omega$.

  Let $\omega\in \Lambda^2(\R^3)$ with components $\omega = \omega_{xy}\df x\w\df y+\omega_{yz}\df y\w\df z+\omega_{zx}\df z\w\df x$, then its exterior derivative yields,
  \begin{align}
    \df \omega = \(\partial_x\omega_{yz}+\partial_y\omega_{zx}+\partial_z\omega_{xy}\)\df x\w\df y\w \df z,
  \end{align}
  which is the equivalent of the divergence of $\omega$.

  Finally, for $\omega\in\Lambda^3(\R^3)$, then $\df\omega=0$ due to the nilpotency of $\df$.
\end{WEbox}

\begin{Ebox}
  \begin{itemize}
    \item Show by explicit calculation that
    \begin{align}
      \df\omega_{(1)}(X,Y) = X[\omega(Y)]-Y[\omega(X)]-\omega(\comm{X}{Y}).
    \end{align}
    \item Show that
    \begin{align}
      f^*\(\df\omega\) =\df\(f^*\omega\).
    \end{align}
    \item Show that
    \begin{align}
      f^*\(\omega\w\xi\) = f^*\omega\w f^*\xi.
    \end{align}
  \end{itemize}
\end{Ebox}


\begin{WEbox}[frametitle={Cohomology},
  frametitlerule=true,
  frametitlealignment=\centering,
  frametitleaboveskip=10pt,]
  The exterior derivative operator induces a sequence, {\it a.k.a.} a complex,
  \begin{align}
    0\stackrel{~i~}{\hookrightarrow }\Lambda^0(M) \xrightarrow{~\df~}\Lambda^1(M) \xrightarrow{~\df~}\cdots\xrightarrow{~\df~} \Lambda^{m-1}(M) \xrightarrow{~\df~}\Lambda^m(M) \xrightarrow{~\df~} 0,
  \end{align}
  called the {\bf de Rham complex}.

  Defining,
  \begin{itemize}
  \item A form, $\omega$, is said to be closed if $\df\omega =0$.
  \item A form, $\omega$, is said to be exact if $\omega =\df\xi$.
  \end{itemize}

  Due to the nilpotency of $\df$, it follows that,
  {\small
  \begin{align*}
    \Im\(\df:C^\infty\(\Lambda^{p-1}(M)\)\to C^\infty\(\Lambda^{p}(M)\)\)\subset\Ker\(\df:C^\infty\(\Lambda^{p}(M)\)\to C^\infty\(\Lambda^{p+1}(M)\) \).
  \end{align*}
  }

  Therefore, the {\bf $\mathbf{p}$-th de Rham Cohomology group}, $H^p(M)$, is defined by
  \begin{align*}
    H^p(M)\equiv \frac{\Ker\(\df:C^\infty\(\Lambda^{p}(M)\)\to C^\infty\(\Lambda^{p+1}(M)\) \)}{\Im\(\df:C^\infty\(\Lambda^{p-1}(M)\)\to C^\infty\(\Lambda^{p}(M)\)\)}.
  \end{align*}
\end{WEbox}


\begin{WEbox}[frametitle={Gauge Theory (Abelian)},
  frametitlerule=true,
  frametitlealignment=\centering,
  frametitleaboveskip=10pt,]
  Let $M$ be a four-dimensional, flat, Lorentzian spacetime endowed with a Minkowski metric, $\eta=\diag(-1,1,1,1)$, and $A_\mu = (-\phi,\vec{A})$ be a four-vector on $T^*M$. One can construct the 1-form 
  \begin{align}
    A = -\phi\df t +A_i\df x^i.
  \end{align}

  Define the 2-form, $F$ as the exterior derivative of $A$,
  \begin{align}
    F &= \df A\\
    &= -\partial_i\phi\;\df x^i\w\df t +\partial_t A_i\;\df t\w\df x^i + \partial_i A_j\;\df x^i\w\df x^j\\
    &= \(\partial_i\phi+\partial_t A_i\)\;\df t\w\df x^i+ \partial_i A_j\;\df x^i\w\df x^j\\
    &= -\df t\w \mathbf{E}_{(1)} +\mathbf{B}_{(2)},
  \end{align}
  where the differential forms $\mathbf{E}$ and $\mathbf{B}$ are the associated to the electric and magnetic field.
  
  Now, the nilpotency of $\df$ implies that,
  \begin{align}
    \df F = \df\df A =0,
  \end{align}
  while on the other hand, 
  \begin{align}
    \df F = \df t\w \df_s\mathbf{E}_{(1)} +\df_t\mathbf{B}_{(2)}+\df_s\mathbf{B}_{(2)},\label{MaxdF}
  \end{align}
  where $\df_t$ and $\df_s$ are the exterior derivative on the time and space respectively.
  
  Since the basis elements $\df t$ and $\df x^i$ are linearly independent, it follows that components containing projections along the time are independent of those lying only on the space. Thus, Eq. (\ref{MaxdF}), decompose into two independent equations,
  \begin{align}
    \partial_t \vec{B} +\vec{\nabla}\cdot\vec{E} &=0,\notag\\
    \vec{\nabla}\cdot\vec{B}&=0,\label{MaxBianchi}
  \end{align}
  where the relation between the exterior derivative on forms and the vector calculus have been used. The Eq. (\ref{MaxBianchi}) are known as the Maxwell equations coming from the Bianchi identity.

  When a change can be made to the geometrical object, $A$ in this case, and the physical fields (what can be measured) do not vary, $\vec{E}$ and $\vec{B}$, the theory is said to posses a  {\bf gauge  invariance}.

  The electromagnetic theory is a gauge theory. Note that the electric and magnetic fields enters through $F$, not through $A$. Thus, if one changes $A\mapsto A+\df f$, the $F$ field do not changes,
  \begin{align}
    F\to F' = \df A' = \df\( A+\df f\) = \df A = F.
  \end{align}
\end{WEbox}


\section{Lie Groups and Lie Algebras}

\subsection{Lie Groups}

\begin{Def}[Group]
  A {\sc Group} $G$ is a set of elements, $\{g\}$, together with an operator, $\cdot: G\to G$, satisfying
  \begin{enumerate}
  \item Exists an unique identity element, $e$, s.t. $e\cdot g=g\cdot e =g$ for all $g\in G$.
  \item For every pair $g_1,g_2\in G$, the product $g_1\cdot g_2\equiv g_3$ belongs to $G$.
  \item Associativity: $g_1\cdot(g_2\cdot g_3)= (g_1\cdot g_2)\cdot g_3$, for all $g_1,g_2,g_3\in G$.
  \item Exists an inverse $g^{-1}\in G\; \forall g\in G$ s.t. $g\cdot g^{-1}=g^{-1}\cdot g=e$.
  \end{enumerate}
\end{Def}



%%%%%%%%%%%%%%%%%% BUNDLES AND CONNECTION

\chapter{Bunbles and Connections}

So far, tangent and cotangent bundles have been considered, and other generalisations have been slightly appearing.

Instead of attacking the problem formally, in the following, a heuristic introduction to more general bundles is shown, and also the definition of connections on these bundles.

\section{Fibre Bundles}

A bundle is a geometric structure composed by two manifolds $E$ (the bundle) and $M$ (the base)\footnote{In some sense the base manifold is a ``piece'' of the bundle, i.e., $M\subset E$.} together with a projection map $\pi:E\to M$. If $U\subset M$, then locally, $E\simeq U\times F$, where $F$ is a manifold called the fibre of $E$. 

\subsubsection*{Examples}
\begin{itemize}
\item A cylinder is a bundle $E$ with base manifold $M=\R$ and fibre $F=S^1$, s.t. $\pi:\R\times S^1\to \R$.

In this case, $E=M\times F$ globally, then the bundle is called {\sc Trivial}.
\item If one takes an interval $I\subset \R$, the ``finite'' cylinder is a trivial bundle $E=I\times S^1$ or $E=S^1\times I$.
\item The simplest non-trivial bundle is a M\"obius strip, which is a bundle $E$ locally homeomorphic to $(I\subset S^1)\times S^1$, but is not globally $S^1\times S^1$.
\item  A 2-torus $T^2$ is a trivial bundle with base manifold $M=S^1$ and fibre $F=S^1$.
\end{itemize}

In essence there are three types of bundles:
\begin{description}
\item [Fibre Bundle] is a general bundle structure, i.e., a manifold which locally looks like a product of a base manifold and the fibre.
\item [Vector Bundle] A fibre bundle whose fibre is a vector space.
\item [Principal Bundle] A fibre bundle whose fibre is a manifold with group structure, i.e., the fibre is a lie group manifold.
\end{description}

\subsection{Sections on Bundles}

A section, $\psi$, on $E\xrightarrow{\pi}M$ is a smooth map from the base manifold to the fibre,
\begin{align}
  \psi:M\to F,
\end{align}
such that $\pi\circ\psi= \id_M$.

Sections are one important object for physicist, because they represent the physical fields of a theory.

\subsection{More About Principal Bundles}

As stated before, principal bundles are fibre bundles whose fibres are a Lie Group. Therefore, there is an action of $G$ on $F$. They are the natural framework for the study of Gauge Theories.

\subsubsection*{Associated Bundles}

The associated bundles are constructed from  principal bundles. Let $P(M,G)$ be a principal bundle, with fibre $F$ admitting and action of the Lie group $G$.

Given a pair $(u,f)$ on $P\times F$, a bundle $\(P\times F\)/G$ can be construct by identifying points related by the action of the group $G$, 
\begin{align}
  (u,f)\sim \(u g,g^{-1}f\).
\end{align}
This construction is known as {\sc Associated Bundle}.

More generally, the fibre can admit the action of a representation $\rho$ of $G$. In this case, the associated bundle $\(P\times_\rho F\)/G$ is constructed through,
\begin{align}
  (u,f)\sim \(u g,\rho\(g^{-1}\)f\).
\end{align}


\section{Connections on the Tangent Bundle}

A connection on the tangent bundle is an application $\nabla:TM\times TM\to TM$, defined by
\begin{align}
  \nabla:(X,Y)\to \nabla_X Y,
\end{align}
satisfying 
\begin{align}
  \nabla_{fX}Y &=f \nabla_X Y\\
  \nabla_X fY &= f\nabla_X Y+ Y . X[f],\\
  \nabla_{X+ Y} Z &= \nabla_X Z+\nabla_Y Z.
\end{align}

On a vector basis, 
\begin{align}
  \nabla_{\partial_i}\partial_j \equiv \nabla_i\partial_j = \conn{i}{k}{j}\pa{k},
\end{align}
then, in components
\begin{align}
  \nab{i}Y=\(\pa{i}Y^j+\conn{i}{j}{k}Y^k\)\pa{j}.
\end{align}
In physics, the term between the brackets is called the covariant derivative of a vector.


\begin{infobox}[frametitle={General Connections}]
  In a general bundle $E$, the connections are applications
  \begin{align*}
    \nabla:TM\otimes\mathcal{E}\to\mathcal{E},
  \end{align*}
  where $\mathcal{E}$ is the space of sections on the bundle $E$.
\end{infobox}

\section{Parallel Transport and Geodesic}

Let $V$ be a tangent vector to a curve, i.e., 
\begin{align}
  V=\frac{d x^\mu}{d t}\(c(t)\)\;\left.\pa{\mu}\right|_{c(t)},
\end{align}
then $X\in TM$ is said to be parallel transport along $c(t)$ if
\begin{align}
  \nabla_V X =0\quad \forall\;t\in I.
\end{align}

If the tangent vector to a curve ($V$) satisfies the parallel transport condition,
\begin{align}
  \nabla_V V =0,
\end{align}
then the curve $c(t)$ is called a {\sc Geodesic}.

\begin{Ebox}
  \begin{itemize}
  \item Use that
    \begin{align*}
      \nab{i}\pa{j} &=\conn{i}{k}{j}\pa{k}\\
      \nab{i}'\pa{j}' &=\(\Ga_{i}'\)^{k}{}_{j}\pa{k}',
    \end{align*}
    to find the transformation rule of $\Ga$'s.
  \item Find the action of $\nabla$ on 1-forms and rank two tensors.
  \item Write in coordinates the condition of parallel transport and geodesic curve.
  \end{itemize}
\end{Ebox}


\section{Torsion and Curvature}

\begin{align}
  T(x,Y) &= \nab{X}Y-\nab{Y}X-\comm{X}{Y}\\
  R(X,Y,Z) &= \nab{X}\nab{Y}Z -\nab{Y}\nab{X}Z -\nab{\comm{X}{Y}}Z.
\end{align}

\begin{Ebox}
  \begin{itemize}
  \item Find coordinate expressions for $T(X,Y)$ and $R(X,Y,Z)$.
  \item Show that they are multilinear objects, i.e., they are tensors.
  \end{itemize}
\end{Ebox}
\bigskip
\begin{infobox}[frametitle={NOTE}]
  In general the concept of curvature can be extended to sections on a bundle, where 
  \begin{align*}
    \nabla:TM\otimes\mathcal{E}\to\mathcal{E},
  \end{align*}
  while the torsion cannot, since $X$ and $Y$ are necessarily the same kind of object.
\end{infobox}



%%%%%%%%%%%%%%%%%% RIEMANNIAN MANIFOLDS

\chapter{Riemannian Manifolds}

\section{The metric}

\begin{Def}[Riemannian Metric]
  Let $M$ be a differential manifold. A {\sc Riemannian metric}, $g$ on $M$ is a symmetric, non-degenerated $\binom{0}{2}$-tensor, i.e.,
  \begin{align}
    g(X,Y) &= g(Y,X)\\
    g(X,X) &= 0\quad\text{iff }X=0.\label{riem-cond}
  \end{align}
\end{Def}

If the metric satisfies the condition
\begin{align}
  g(X,Y) = 0\quad \forall X\;\Rightarrow\; Y=0.\label{sriem-cond}
\end{align}
instead of Eq. (\ref{riem-cond}), the metric is said to be semi-Riemannian.

Given a set of coordinated on $M$, the metric is expressed as 
\begin{align}
  g_p = g_{\mu\nu}(p)\;\df x^\mu\otimes\df x^\nu,
\end{align}
with
\begin{align}
  g_{\mu\nu}(p) = g_p\(\partial_\mu,\partial_\nu\).
\end{align}

The metric can be seen as a map
\begin{align}
  g_p:T_pM\times T_pM &\to \R,\\
  g_p:T_pM &\to T^*_pM,
\end{align}
and its inverse, $g^{-1}$, similarly yields
\begin{align}
  g^{-1}_p:T^*_pM\times T^*_pM &\to \R,\\
  g^{-1}_p:T^*_pM &\to T_pM.
\end{align}

In index notation, the inverse metric is denoted with upper indices, i.e., 
\begin{align}
  \(g^{-1}\)_{\mu\nu} = g^{\mu\nu},
\end{align}
such that,
\begin{align}
  g_{\mu\nu}g^{\nu\lambda} = \delta_\mu^\lambda.
\end{align}

One might consider metrics compatible with the connection, 
\begin{align}
  \nabla_X g = 0,\quad \forall\,X\in TM.
\end{align}

\begin{Ebox}
  Use the {\bf metric compatibility} condition, to find a expression for the (general) connection in terms of the metric and the skew-symmetric part of the $\Gamma$'s.
\end{Ebox}


\section{Metric Structure and Impact on Differential Forms}

It has been pointed out before that the only geometrical object one can integrate on an orientable $m$-dimensional manifold is a $m$-form. However, on Riemannian manifolds, were the metric defines a volume form, one in accustomed to integrate functions.

The link between the two notions is made by defining a new sort of differential operator, called the {\bf Hodge star}, as follows
\begin{align}
  *:C^\infty\(\Lambda^p(M)\)\to C^\infty\(\Lambda^{m-p}(M)\)
\end{align}
such that, for $\alpha,\beta\in C^\infty\(\Lambda^p(M)\)$,
\begin{align}
  \bk{\alpha}{\beta}=\int \alpha\w *\beta= \int_M (\alpha,\beta)dV_g = \int\,\dn{m}{x}\sqrt{g}\,\alpha_{a_1\cdots a_p}\beta_{b_1\cdots b_p}g^{a_1 b_1}\cdots g^{a_p b_p}.\label{HodgeDef}
\end{align}


\begin{Ebox}
  \begin{itemize}
  \item Use Eq. (\ref{HodgeDef}) to find the action of $*$ on the basis of differential forms.
  \item Show that the double action of the Hodge star is proportional to the unit, i.e.,
    \begin{align*}
      *^2\alpha \simeq \alpha
    \end{align*}
  \end{itemize}
\end{Ebox}
      

Using this, a formal  dual of the differential operator $\df$ can be defined,
\begin{align}
  \dfd:\Lambda^p(M)\to\Lambda^{p-1}(M),
\end{align}
such that (on compact manifolds)
\begin{align}
  \bk{\dfd \alpha_{(p+1)}}{\beta_{(p)}}=\bk{\alpha_{(p+1)}}{\df\beta_{(p)}}.
\end{align}
The operator $\dfd$ is called co-differential, and it is nilpotent as its partner $\df$, i.e., $\df^2 ={\dfd}^2=0$. Also, a differential form is said to be co-closed is $\dfd\alpha=0$ and co-exact if $\alpha=\dfd\beta$, thus, $\dfd$ defines a complex, and a homology.


\begin{Ebox}
  Use the symmetric properties of the product of forms and the ``idempotence'' of the Hodge star operation, to find an expression of the co-differential operator in terms of $\df$ and $*$.
\end{Ebox}
     


Using the differential and co-differential operators on forms, the Laplacian on differential forms can be defined by,
\begin{align}
  \Delta:C^\infty\(\Lambda^p(M)\) &\to C^\infty\(\Lambda^p(M)\)\\
  \Delta&= \df\dfd+\dfd\df
\end{align}
A form satisfying $\Delta\alpha=0$ is said to be an {\bf Harmonic Form}.

\begin{Thm}
  On a compact, oriented Riemannian manifold one can define the space of harmonic forms by
  \begin{align}
    \Ha^k = \Ker\(\Delta:C^\infty\(\Lambda^p(M)\) \to C^\infty\(\Lambda^p(M)\)\),
  \end{align}
  then it follows that for every $\alpha\in \Ha^k$, $\alpha$ is closed and co-closed. 
\end{Thm}
\begin{proof}
  Since $\Delta\alpha=0$, then
  \begin{align}
    0=\bk{\alpha}{\Delta\alpha}&=\bk{\alpha}{\df\dfd\alpha}+ \bk{\alpha}{\dfd\df\alpha}\\
    &=\bk{\dfd\alpha}{\dfd\alpha}+ \bk{\df\alpha}{\df\alpha}\\
    &= \norm{\dfd\alpha}^2+ \norm{\df\alpha}^2.
  \end{align}
  Since the sum of positive quantities vanishes, each quantity must vanish separately. Therefore,
  \begin{align}
    \df\alpha=\dfd\alpha=0.
  \end{align}
\end{proof}


\section{Defining Actions in Physics}

Now the machinery for constructing physical actions has been presented. Thus in this section, different actions will be shown.

%In the lectures a part of the Abelian gauge theories was worked out, therefore it will be presented as first example.

\subsection{Classical (non-relativistic) Point Particle}

In classical physics the observed quantities are coordinates in $\R^3$, $x^i$ with $i=1,2,3$, denoting the position of a particle, while the only parameter is the time,  $t\in \R$.

Therefore, a classical, non-relativistic, massive, point-particle is described by a map $x:\R\to\R^3$. In the language of bundles, $x$ is a section on a fibre bundle, $E\xrightarrow{~\pi~}M$, where the base manifold is the $\R$ (the time manifold), and its fibre is $\R^3$. Additionally, on the fibre an inner product is defined, $\bk{\bullet}{\bullet}$.




\subsection{Electromagnetic Action}

The electromagnetic action is a theory of a connection $\Af{1}$, with values on an Abelian gauge group ($U(1)$). The fundamental object for constructing the action is the field strength, $\FF{2}=\df\Af{1}$, which is gauge invariant.

Using differential forms the action is,
\begin{align}
  \int\Lag[\Af{1}] = -\frac{1}{2}\int \FF{2}\w *\FF{2}+\int \Af{1}\w*J
\end{align}


\begin{Ebox}
  \begin{itemize}
  \item Use the properties of the exterior differentiation to find the equation of motion of the field $\Af{1}$.
  \item Write the equations in components, then use the definitions of $\vec{E}$ and $\vec{B}$ to find the usual Maxwell equations.
  \item Use the nilpotence of $\df$ (or $\dfd$) to find the continuity equation of $J$.
  \item Show that the Lorentz condition $\partial_\mu A^\mu$ is expressed as $\df*\Af{1}$.
  \end{itemize}
\end{Ebox}
     


\subsection{Non-Abelian Gauge Theories}

In order to go beyond the Abelian gauge theories, it is useful to give geometrical interpretation to the fields.

As all of the physical theories, a gauge theory lies on a Minkowski spacetime, $M$, and the fields transform as irreps of the Lorentz group. In the case of a gauge boson,  $\Af{1}$, the field transforms under the (co)vector representation, therefore $\Af{1}$ is a section on the cotangent bundle ($T^*M$).

Additionally, the field transforms as a connection under gauge transformations of a group $G$. Then, the field is also a section of the principal bundle $P(M,G)$, specifically $\Af{1}$ is a section on the associated bundle $P(M,G)\times_{Ad_G}\frak{g}$.

Since the field is not invariant, the Lagrangian density must be made invariant. Then, an inner product on $\frak{g}$ should be considered, denoted $\vev{\bullet}$.

Finally, the Lagrangian density is
\begin{align}
  \Lag[\Af{1}] =-\frac{1}{2} \vev{\FF{2}\w*\FF{2}},
\end{align}
where $\FF{2}$ is the curvature of the connection $\Af{1}$, defined by the ``twisted'' exterior derivative $\df_{\Af{1}}= \df +\Af{1}$,
\begin{align}
  \FF{2}= \df_{\Af{1}}\Af{1}= \df\Af{1} +\frac{1}{2}\Af{1}\w\Af{1}
\end{align}


\begin{WEbox}[frametitle={Equations of Motion for Yang-Mills Theories},
  frametitlerule=true,
  frametitlealignment=\centering,
  frametitleaboveskip=10pt,]
  In order to find the equation of motion of a Yang-Mills theory, one might use the fact that the ``twist'' derivative, $\df_{\Af{1}}$, acts on charged fields, then
  \begin{align*}
    \int \Lag[\Af{1}] &=-\frac{1}{2}\int \vev{\FF{2}\w*\FF{2}}\\
    &=-\frac{1}{2}\int \df_{\Af{1}}\vev{\Af{1}\w*\FF{2}}-\frac{1}{2}\int \vev{\Af{1}\w\df_{\Af{1}}*\FF{2}}\\
    &=-\frac{1}{2}\int \df\vev{\Af{1}\w*\FF{2}}-\frac{1}{2}\int \vev{\Af{1}\w\df_{\Af{1}}*\FF{2}},
  \end{align*}
  the first term vanishes for fields with compact support (or manifolds without boundaries), then, the equation of motion is 
  \begin{align*}
    \dfd_{\Af{1}}\FF{2} =0.
  \end{align*}
\end{WEbox}
   




\section{General Relativity Tensors}

The formulation of General Relativity (GR) lies on the absence of torsion, therefore the only geometrical tensor which measures the lack of Euclidean structures on $M$ is the curvature.

Roughly speaking the goal of GR is to associate the gravitational interaction to the geometrical deformation of the manifold. Therefore, one must mix objects with the geometrical information of the manifold with the object containing the matter distribution. The former is related with the Riemann tensor, while the later is the stress-energy tensor. Immediately, one notes that the rank of these tensors is not compatible, thus some arrange must be done.

Using the curvature tensor, other simplified tensors can be constructed
\begin{align}
  \Ric(X,Y) &= \bk{\df x^\mu}{R(\partial_\mu,Y)X}\\
  \Ri &= g^{\mu \nu}\Ric(\partial_\mu,\partial_\nu).
\end{align}
Since in most cases the stress-energy tensor is symmetric, it seems that Einstein's first proposal for the equations of gravity were
\begin{align}
  \Ri_{\mu\nu}=T_{\mu\nu},
\end{align}
but the R.H.S. of the equation was covariantly constant, while the L.H.S. was not. Thus, a covariantly constant construction made up with ``curvature'' tensors was
\begin{align}
  G_{\mu\nu} = \Ri_{\mu\nu}-\frac{1}{2}g_{\mu\nu}\Ri,
\end{align}
this tensor is known as Einstein tensor, ans the equation of gravity are written as,
\begin{align}
  G_{\mu\nu}= T_{\mu\nu}.
\end{align}
Additionally, there are spaces with ``cosmological constant'', $\Lambda$, and the Einstein's equations for this kind of spacetimes are
\begin{align}
  \Ri_{\mu\nu}-\frac{1}{2}g_{\mu\nu}\Ri + \Lambda g_{\mu\nu}= T_{\mu\nu}.
\end{align}

A manifold whose Ricci tensor is proportional to the metric is said to be an Einstein manifold,
\begin{align}
  \Ri_{\mu\nu} = \kappa g_{\mu\nu}.
\end{align}

Also, a manifold is said to have constant curvature if its curvature 2-form is
\begin{align}
  \Rif{ab}{} = \kappa\vif{a}\w\vif{b},
\end{align}
for a constant value $\kappa$.



%%%%%%%%%%%%%%%%%% ANOTHER LECTURE

\chapter{Cartan Calculations in (semi)Riemannian Geometry}

Before continuing, a change of notation will be made. Although this can be confusing, the previous notation is used often by mathematicians, while the following is used by physicist.

\section{Frame Fields in (semi)Riemannian Geometry}

%Before continuing, a change of notation will be made. Although this can be confusing, the previous notation is used often by mathematicians, while the following is used by physicist.

Let $M$ be a manifold, and $g$ a (semi)Riemannian metric defined on $M$. Then the line element for the metric $g$ is
\begin{align}
  ds^2(g) = g_{\mu\nu}\df x^\mu \otimes\df x^\nu.
\end{align}
Nonetheless, the information about the metric structure of the manifold can be translate to the languages of frames,
\begin{align}
  ds^2(g) &= g_{\mu\nu}\df x^\mu \otimes\df x^\nu\\
  &= \eta_{ab} \vi{a}{\mu}\vi{b}{\nu}\df x^\mu \otimes\df x^\nu\\
  &=\eta_{ab}\vif{a}\otimes\vif{b}.
\end{align}
Therefore, the vielbein, $\vif{a}$, is the fields equivalent to the metric tensor. In order to complete the structure, one needs information about the transport of geometrical objects lying on bundles based on $M$. That information is encoded on the spin connection, $\spif{a}{b}$. Using these quantities one finds  the generalisation of the structure equations of Cartan,
\begin{align}
  \df\vif{a} +\spif{a}{b}\w\vif{b} &= \Tf{a},\label{firstSE}\\
  \df\spif{a}{c}+\spif{a}{b}\w\spif{b}{c} &= \Rif{a}{c}\label{secondSE}.
\end{align}
In the sense of the previous lesson, the torsion 2-form, $\Tf{a}$, and the curvature 2-form,$\Rif{a}{c}$, measure the impossibility of endow an Euclidean structure on $M$.

\subsection*{Remarks on the Structure Equations}

\begin{itemize}
\item In general relativity, the torsion vanishes. Therefore, the first Cartan's structure equation (\ref{firstSE}), yields an explicit expression for the spin connection as a function of the vielbein. This is equivalent to the well-known expression of the Levi-Civita connection $\Gamma_\mu{}^\lambda{}_\nu$ as a function of the metric and its derivatives.
\item The components of the torsion and curvature 2-forms are given by
  \begin{align}
    \tor_\mu{}^\lambda{}_\nu \;\df x^\mu\w\df x^\nu &= \tor_a{}^c{}_b \vin{\lambda}{c}\;\vif{a}\w\vif{b},\\
    \Rif{a}{b} &= \frac{1}{2}\Ri^a{}_{b\mu\nu}\;\df x^\mu\w\df x^\nu.
  \end{align}
\end{itemize}

\begin{WEbox}
  The curvature of the Anti de Sitter ($AdS$) spacetime is calculated, using Cartan's formalism.

  The metric of the $AdS$ spacetime is 
  \begin{align}
    ds^2(g) = \frac{1}{z^2}\(-\df t\otimes\df t +\df x\otimes\df x+\df y\otimes\df y+\df z\otimes\df z\),
  \end{align}
  then the vielbeins are
  \begin{align}
    \vif{0} &= \frac{\df t}{z}, & \vif{1} &= \frac{\df x}{z},\notag\\
    \vif{2} &= \frac{\df y}{z}, & \vif{3} &= \frac{\df z}{z},
  \end{align}
  then,
  \begin{align}
    \df\vif{0}&=-\frac{1}{z^2}\df z\w\df t & \df\vif{1}&=-\frac{1}{z^2}\df z\w\df x\notag\\
    &= \frac{1}{z}\df t\w\vif{3} & &= \frac{1}{z}\df x\w\vif{3}\\
    \df\vif{2}&=-\frac{1}{z^2}\df z\w\df y & \df\vif{3}&=0\notag \\
    &= \frac{1}{z}\df y\w\vif{3} & & \notag
  \end{align}
  therefore,
  \begin{align}
    \spif{0}{3} = -\frac{1}{z}\df t,\quad \spif{1}{3}=-\frac{1}{z}\df x,\quad \spif{2}{3} = -\frac{1}{z}\df y.
  \end{align}
  Now, using the second structure equation, Eq. (\ref{secondSE}), it follows that
  \begin{align}
    \Rif{0}{3} &= \df\spif{0}{3} = \frac{1}{z^2}\df z\w\df t &  \Rif{1}{3} &= \df\spif{1}{3} = \frac{1}{z^2}\df z\w\df x \notag\\
    \Rif{2}{3} &= \df\spif{2}{3} = \frac{1}{z^2}\df z\w\df y &  \Rif{0}{1} &= \spif{0}{3}\w\spif{3}{1} = -\frac{1}{z^2}\df t\w\df x \\
    \Rif{0}{2} &= \spif{0}{3}\w\spif{3}{2} = -\frac{1}{z^2}\df t\w\df y & \Rif{1}{2} &= \spif{1}{3}\w\spif{3}{2} = -\frac{1}{z^2}\df x\w\df y.\notag
  \end{align}
  
  Now, the non-vanishing components of the Riemann tensor are,
  \begin{align}
    \Ri^t{}_{ztz} &= -\frac{1}{z^2} & \Ri^x{}_{zxz} &= -\frac{1}{z^2}\notag\\
    \Ri^y{}_{zyz} &= -\frac{1}{z^2} & \Ri^t{}_{xtx} &= -\frac{1}{z^2}\\
    \Ri^t{}_{yty} &= -\frac{1}{z^2} & \Ri^x{}_{yxy} &= -\frac{1}{z^2}\notag\\
  \end{align}
  and finally, the Ricci tensor is
  \begin{align}
    \Ri_{\mu\nu} = -3 g_{\mu\nu}.
  \end{align}

  The $AdS$ spacetime is an example of an Einstein space, or in other words it is a solution of Einstein's equations with cosmological constant, $\Lambda=-3$.
\end{WEbox}

\begin{Ebox}
  Consider a Riemannian manifold with a metric whose line element is
  \begin{align}
    ds^2(g) = -f(r)\df t\otimes \df t+ \frac{1}{f(r)}\df r\otimes\df r+ r^2\(\df\theta\otimes\df\theta+\sin^2(\theta)\df\phi\otimes\df\phi\),
  \end{align}
  Find the function $f(r)$ which solves the Einstein's equation in the vacuum.
\end{Ebox}

\section{Playing with the Structure Equations}

From the first S.E.
\begin{align*}
  \df\(\df\vif{a}+\spif{a}{b}\w\vif{b}\) =& \df\Tf{a}\\
  \therefore\quad \(\df\spif{a}{b}\)\w\vif{b}-\spif{a}{b}\w\df\vif{b} =& \df\Tf{a}\\
  \(\Rif{a}{b}-\spif{a}{c}\w\spif{c}{b}\)\w\vif{b}-\spif{a}{c}\w\df\vif{c}=&\df\Tf{a}\\
  \Rif{a}{b}\w\vif{b}-\spif{a}{c}\w\(\spif{c}{b}\w\vif{b}+\df\vif{c}\)=&\df\Tf{a}\\
  \Rif{a}{b}\w\vif{b}-\spif{a}{c}\w\Tf{c}=&\df\Tf{a}\\
  \Rif{a}{b}\w\vif{b}=&\df\Tf{a}+\spif{a}{c}\w\Tf{c}\\
  \Rif{a}{b}\w\vif{b}=&\cdf\Tf{a}
\end{align*}


From the second S.E.
\begin{align*}
  \df\(\df\spif{a}{b}+\spif{a}{c}\w\spif{c}{b}\) =& \df\Rif{a}{b}\\
  \therefore\quad \(\df\spif{a}{c}\)\w\spif{c}{b}-\spif{a}{c}\w\(\df\spif{c}{b}\) =& \df\Rif{a}{b}\\
  \Rif{a}{c}\w\spif{c}{b}-\spif{a}{c}\w\Rif{c}{b} =& \df\Rif{a}{b}\\
  \df\Rif{a}{b} + \spif{a}{c}\w\Rif{c}{b} - \Rif{a}{c}\w\spif{c}{b}   =& 0,
\end{align*}

