%%%%%%%%%%%%%%%%%% DIFFERENTIAL FORMS
\chapter{Exterior Calculus}

\section{Differential Forms}

\begin{Def}[Differential Forms]
  A \textsc{\idx{Differential Form}} of order $p$, or $p$-form, is a totally antisymmetric tensor of type $\binom{0}{p}$.
\end{Def}

Since differential form are tensor, the basis could be the same than for tensors, i.e.,
\begin{align*}
  \set{\pa{\mu}},\set{\pa{\mu}\otimes\pa{\nu}},\set{\pa{\mu}\otimes\pa{\nu}\otimes\pa{\lambda}},\cdots,
\end{align*}
however, since forms are totally antisymmetric, it is more appealing to define an antisymmetric product of the basis. This new product is called ``wedge'' product ($\we$), and is defined by
\begin{align}
  \de{x}^\mu\we\de{x}^\nu &= \de{x}^\mu\otimes\de{x}^\nu-\de{x}^\nu\otimes \de{x}^\mu\\
 &\vdots\\
  \de{x}^{\mu_1}\we\cdots\we\de{x}^{\mu_p}&= \sum_{\text{perm.}}(-1)^{\abs{\sigma}}\de{x}^{\sigma(\mu_1)}\otimes\cdots\otimes\de{x}^{\sigma(\mu_p)}
\end{align}

The vector space of differential forms of order $p$ on a manifold $M$ is denoted by $\Omega^p(M)$ or $\Lambda^p(M)$. Moreover, if $\dim(M)=m$, then
\begin{align}
  \dim\(\Lambda^p(M)\)=\binom{m}{p} =\frac{m!}{p!(m-p)!}.
\end{align}

\begin{infobox}
  In general $\Lambda^p(M)\subset \otimes^p T^*M$.
\end{infobox}

\subsection[Exterior Product]{Exterior Product (or wedge product)}

The exterior product is a map 
\begin{align}
  \we:\Lambda^p(M)\times\Lambda^q(M) \to \Lambda^{p+q}(M)
\end{align}
defined by
\begin{align}
  (\omega\we\xi)\(V_1,\cdots,V_{p+q}\) = \frac{1}{p!q!}\sum_{\text{perm.}}(-1)^{\abs{\sigma}}\omega\(V_{\sigma(1)},\cdot,V_{\sigma(p)}\)\xi\(V_{\sigma(p+1)},\cdot,V_{\sigma(p+q)}\)
\end{align}

This exterior product endows an algebra structure to 
\begin{align}
  \Lambda^\bullet(M) \equiv \Lambda^0(M)\oplus\Lambda^1(M)\oplus\cdots\oplus \Lambda^m(M).
\end{align}
Clearly,
\begin{align}
  \dim\(\Lambda^\bullet(M)\) = 2^m.
\end{align}

\subsection{Exterior Derivative}

As stated previously, although the derivative transforms like a 1-form, the derivative of a tensor is not necessarily a tensor. However, it is possible to define a ``covariant'' derivative which maps tensors into tensors.

In the same way, one defines an exterior derivative which maps $p$-forms into $(p+1)$-forms,
\begin{align}
  \df\, : \Lambda^p(M)\to \Lambda^{p+1}(M),
\end{align}
defined as follows. Let $\omega$ be a differential $p$-form on $M$, then
\begin{align}
  \df\,\omega\(X_1,\cdots,X_{p+1}\) &= \sum_{i=1}^p (-1)^{i+1}X_i\[\omega\(X_1,\cdots,\hat{X}_i,\cdots,X_{p+1}\)\]\\
  &\phantom{ ==}\sum_{i<j} (-1)^{i+j}\omega\(\comm{X_i}{X_j},X_1,\cdots,\hat{X}_i,\cdots\hat{X}_i,\cdots,X_{p+1}\).\notag
\end{align}

\begin{Ebox}
  \begin{itemize}
  \item Show by explicit calculation that
    \begin{align}
      \df\,\omega_{(1)}(X,Y) = X[\omega(Y)]-Y[\omega(X)]-\omega(\comm{X}{Y}).
    \end{align}
  \item Show that if $\omega$ is a differential $p$-form with components
    \begin{align}
      \omega = \frac{1}{p!}\omega_{\mu_1\cdots\mu_p}\;\de{x}^{\mu_1}\we\cdots\we\de{x}^{\mu_p},
    \end{align}
    then its exterior derivative yields,
    \begin{align}
      \df\,\omega = \frac{1}{p!}\pa{\nu}\omega_{\mu_1\cdots\mu_p}\;\de{x}^\nu\we\de{x}^{\mu_1}\we\cdots\we\de{x}^{\mu_p}.
    \end{align}
  \end{itemize}
\end{Ebox}


It follows from the definition of the exterior derivative, that the double action of $\df\,$ vanishes, i.e., $\df\,^2=0$. This property is known as {\bf nilpotency}.





\begin{WEbox}[frametitle={Differential Forms in $\R^3$},
  frametitlerule=true,
  frametitlealignment=\centering,
  frametitleaboveskip=10pt,]
  In three dimensions the maximum order of a differential form is 3. Therefore, the complete set of forms on $\R^3$ is given by
  \begin{align*}
    \Lambda^0(\R^3) &=\F(\R^3), & \Lambda^1(\R^3) &= T^*\R^3\simeq \R^3,\notag\\
    \Lambda^2(\R^3) &= T^*\R^3\we T^*\R^3\simeq \R^3, & \Lambda^3(\R^3) &=T^*\R^3\we T^*\R^3\we T^*\R^3\simeq\F(\R^3).
  \end{align*}
  
  Let $f\in\Lambda^0(\R^3) $, then the exterior derivative of $f$ yields
  \begin{align}
    \df\, f= \frac{\partial f}{\partial x}\de{x} +\frac{\partial f}{\partial y}\df\, y +\frac{\partial f}{\partial z}\df\, z,
  \end{align}
  which is the equivalent of the gradient operation.
  
  Let $\omega\in \Lambda^1(\R^3)$ with components $\omega = \omega_x\de{x}+\omega_y\df\, y+\omega_z\df\, z$, then its exterior derivative yields,
  \begin{align}
    \df\,\omega &= \(\partial_x\omega_y-\partial_y\omega_x\)\de{x}\we\df\, y +\(\partial_y\omega_z-\partial_z\omega_y\)\df\, y\we\df\, z\notag\\
    &\phantom{==}+\(\partial_z\omega_x-\partial_x\omega_z\)\df\, z\we\de{x}, 
  \end{align}
  which is the equivalent of the curl of $\omega$.

  Let $\omega\in \Lambda^2(\R^3)$ with components $\omega = \omega_{xy}\de{x}\we\df\, y+\omega_{yz}\df\, y\we\df\, z+\omega_{zx}\df\, z\we\de{x}$, then its exterior derivative yields,
  \begin{align}
    \df\, \omega = \(\partial_x\omega_{yz}+\partial_y\omega_{zx}+\partial_z\omega_{xy}\)\de{x}\we\df\, y\we \df\, z,
  \end{align}
  which is the equivalent of the divergence of $\omega$.

  Finally, for $\omega\in\Lambda^3(\R^3)$, then $\df\,\omega=0$ due to the nilpotency of $\df\,$.


  Now, if one would like to compare with some differential properties on vector calculus, from the nilpotence of $\df\,$, it follows that
  \begin{align}
    \df\,(\df\, f) =0\quad&\Rightarrow \quad \vec{\nabla}\times \vec{\nabla}f =0,\\
    \df\,\(\df\,\omega_{(1)}\)=0 \quad&\Rightarrow\quad  \vec{\nabla}\cdot\( \vec{\nabla}\times \vec{\omega}\)=0.
  \end{align}
\end{WEbox}


\subsection{The Pullback}

A map $f:M\to N$ induces the pullback
\begin{align}
  f^*:T^*_{f(p)}N\to T^*_pM,
\end{align}
which is extended naturally to type $\binom{0}{p}$-tensors (and to differential $p$-forms),
\begin{align}
  (f^*\omega)(V_1,\cdots,V_p) =\omega(f_*V_1,\cdots,f_*V_p).
\end{align}

\begin{Ebox}
  \begin{itemize}
  \item Show that
    \begin{align}
      f^*\(\df\,\omega\) =\df\,\(f^*\omega\).
    \end{align}
    \item Show that
    \begin{align}
      f^*\(\omega\we\xi\) = f^*\omega\we f^*\xi.
    \end{align}
  \end{itemize}
\end{Ebox}


\begin{WEbox}[frametitle={Cohomology},
  frametitlerule=true,
  frametitlealignment=\centering,
  frametitleaboveskip=10pt,]
  The exterior derivative operator induces a sequence, {\it a.k.a.} a complex,
  \begin{align}
    0\stackrel{~i~}{\hookrightarrow }\Lambda^0(M) \xrightarrow{~\df\,~}\Lambda^1(M) \xrightarrow{~\df\,~}\cdots\xrightarrow{~\df\,~} \Lambda^{m-1}(M) \xrightarrow{~\df\,~}\Lambda^m(M) \xrightarrow{~\df\,~} 0,
  \end{align}
  called the {\bf de Rham complex}.

  Defining,
  \begin{itemize}
  \item A form, $\omega$, is said to be closed if $\df\,\omega =0$.
  \item A form, $\omega$, is said to be exact if $\omega =\df\,\xi$.
  \end{itemize}

  Due to the nilpotency of $\df\,$, it follows that,
  {\small
  \begin{align*}
    \Im\(\df\,:C^\infty\(\Lambda^{p-1}(M)\)\to C^\infty\(\Lambda^{p}(M)\)\)\subset\Ker\(\df\,:C^\infty\(\Lambda^{p}(M)\)\to C^\infty\(\Lambda^{p+1}(M)\) \).
  \end{align*}
  }

  Therefore, the {\bf $\mathbf{p}$-th de Rham Cohomology group}, $H^p(M)$, is defined by
  \begin{align*}
    H^p(M)\equiv \frac{\Ker\(\df\,:C^\infty\(\Lambda^{p}(M)\)\to C^\infty\(\Lambda^{p+1}(M)\) \)}{\Im\(\df\,:C^\infty\(\Lambda^{p-1}(M)\)\to C^\infty\(\Lambda^{p}(M)\)\)}.
  \end{align*}
\end{WEbox}


\subsection{Interior Product and Lie Derivative}

The interior product is  a map
\begin{align}
  i:TM\times \Lambda^p(M)\to \Lambda^{p-1}(M),
\end{align}
or usually denoted
\begin{align}
  i_X:\Lambda^p(M)\to \Lambda^{p-1}(M),
\end{align}
defined by
\begin{align}
  i_X\omega(V_1,\cdots,V_{p-1})=\omega(X,V_1,\cdots,V_{p-1}).
\end{align}
In components it can be written as
\begin{align}
  i_X\omega &= \frac{1}{(p-1)!}X^\nu \omega_{\nu \mu_2 \cdots\mu_{p}}\;\de{x}^{\mu_2}\we\cdots\we\de{x}^{\mu_p}\\
  &= \frac{1}{p!}\sum_{s=1}^p X^{\mu_s} \omega_{\mu_1 \cdots\mu_s \cdots \mu_{p}}\;\de{x}^{\mu_2}\we\cdots\we\widehat{\de{x}}^{\mu_s}\we\cdots\we\de{x}^{\mu_p}.\notag
\end{align}


Using the exterior derivative and the interior product, one can construct the operator $\df\, i_x+i_X \df\,$. This operator can be applied to 1-forms,
\begin{align}
  \(\df\, i_x+i_X \df\,\)\omega &=\df\,(X^\mu\omega_\mu) +i_x\(\pa{\mu}\omega_\nu \de{x}^\mu\we\de{x}^\nu\)\notag\\
  &=\(\pa{\nu}X^\mu\)\omega_\mu\de{x}^\nu +X^\mu\(\pa{\nu}\omega_\mu\)\de{x}^\nu +X^\mu\(\pa{\mu}\omega_\nu-\pa{\nu}\omega_\mu\)\de{x}^\nu\notag\\
  &=\[\omega(\pa{\nu}X^\mu)+X^\mu(\pa{\mu}\omega_\nu)\]\de{x}^\nu,
\end{align}
which is nothing but the Lie derivative of a differential 1-form,
\begin{align}
  \Li_X\omega = \(\df\, i_X + i_X \df\,\)\omega.
\end{align}
The above expression is general for differential forms of any order.

The interior product and the Lie derivative satisfy
\begin{align}
  i_X\(\omega\we\eta\)&=\(i_X\omega\) \we \eta+(-1)^{\abs{\omega}}\omega \we \(i_X\eta\)\\
  i_X^2&=0\\
  \Li_X i_X\omega &= i_X\Li_X\omega.
\end{align}


\begin{WEbox}[frametitle={Geometry of Classical Mechanics},
  frametitlerule=true,
  frametitlealignment=\centering,
  frametitleaboveskip=10pt,]
  Let $M$ be a $m$-dimensional manifold, and $\Phi^m\equiv T^*M$ be its cotangent bundle (or {\bf phase space}). A {\bf Hamiltonian} on $M$ is a map $\Ha:\Phi^m\to \R$.

  Let $\set{q^\mu,p_\mu}$, with $\mu=1...m$, be  the coordinates on $\Phi^m$. One might define a symplectic form on $\Phi^m$ by
  \begin{align*}
    \Lambda^2\(\Phi^m\)\ni \omega = \df\, p_\mu\we\df\, q^\mu,
  \end{align*}
  and a 1-form $\theta=q^\mu \df\, p_\mu\in \Lambda^1\(\Phi^m\)$, s.t. $\omega=-\df\,\theta$.
  
  Given a function $f:\Phi^m\to \R$. one might define the {\sc Hamiltonian Vector Field}
  \begin{align*}
    X_f = \pder{f}{p_\mu}\pder{}{q^\mu} -\pder{f}{q^\mu}\pder{}{p_\mu}.
  \end{align*}
  Then,
  \begin{align*}
    i_{X_f}\omega =-\pder{f}{q^\mu}\df\, q^\mu -\pder{f}{p_\mu}\df\, p_\mu = -\df\, f.
  \end{align*}

  The Hamilton equations,
  \begin{align*}
    \dot{q}^\mu =\pder{\Ha}{p_\mu},\quad \dot{p}_\mu =-\pder{\Ha}{q^\mu},
  \end{align*}
  can be written as
  \begin{align*}
    i_{X_\Ha}\omega = -\pder{\Ha}{q^\mu}\df\, q^\mu -\pder{\Ha}{p_\mu}\df\, p_\mu = -\df\,\Ha,
  \end{align*}
  where the identity
  \begin{align*}
    X_\Ha &=\pder{\Ha}{p_\mu}\pder{}{q^\mu} -\pder{\Ha}{q^\mu}\pder{}{p_\mu}\\
    &= \dot{q}^\mu\pder{}{q^\mu} +\dot{p}_\mu\pder{}{p_\mu}\\
    &= \der{}{t},
  \end{align*}
  has been used.

  Also, the action of two interior products on $\omega$ yields the Poisson bracket,
  \begin{align*}
    i_{X_f}\(i_{X_g}\omega\) &=-i_{X_f}(\df\, g) = -i_{X_f}\(\pder{g}{q^\mu}\df\, q^\mu +\pder{g}{p_\mu}\df\, p_\mu\)\\
    &= -\pder{f}{p_\mu}\pder{g}{q^\mu}+\pder{f}{q^\mu}\pder{g}{p_\mu}\\
    &=\acomm{f}{g}_{P.B.}.
  \end{align*}
\end{WEbox}

\begin{WEbox}[frametitle={Gauge Theory (Abelian)},
  frametitlerule=true,
  frametitlealignment=\centering,
  frametitleaboveskip=10pt,]
  Let $M$ be a four-dimensional, flat, Lorentzian spacetime endowed with a Minkowski metric, $\eta=\diag(-1,1,1,1)$, and $A_\mu = (-\phi,\vec{A})$ be a four-vector on $T^*M$. One can construct the 1-form 
  \begin{align}
    A = -\phi\df\, t +A_i\de{x}^i.
  \end{align}

  Define the 2-form, $F$ as the exterior derivative of $A$,
  \begin{align}
    F &= \df\, A\\
    &= -\partial_i\phi\;\de{x}^i\we\df\, t +\partial_t A_i\;\df\, t\we\de{x}^i + \partial_i A_j\;\de{x}^i\we\de{x}^j\\
    &= \(\partial_i\phi+\partial_t A_i\)\;\df\, t\we\de{x}^i+ \partial_i A_j\;\de{x}^i\we\de{x}^j\\
    &= -\df\, t\we \mathbf{E}_{(1)} +\mathbf{B}_{(2)},
  \end{align}
  where the differential forms $\mathbf{E}$ and $\mathbf{B}$ are the associated to the electric and magnetic field.
  
  Now, the nilpotency of $\df\,$ implies that,
  \begin{align}
    \df\, F = \df\,\df\, A =0,
  \end{align}
  while on the other hand, 
  \begin{align}
    \df\, F = \df\, t\we \df\,_s\mathbf{E}_{(1)} +\df\,_t\mathbf{B}_{(2)}+\df\,_s\mathbf{B}_{(2)},\label{MaxdF}
  \end{align}
  where $\df\,_t$ and $\df\,_s$ are the exterior derivative on the time and space respectively.
  
  Since the basis elements $\df\, t$ and $\de{x}^i$ are linearly independent, it follows that components containing projections along the time are independent of those lying only on the space. Thus, Eq. (\ref{MaxdF}), decompose into two independent equations,
  \begin{align}
    \partial_t \vec{B} +\vec{\nabla}\cdot\vec{E} &=0,\notag\\
    \vec{\nabla}\cdot\vec{B}&=0,\label{MaxBianchi}
  \end{align}
  where the relation between the exterior derivative on forms and the vector calculus have been used. The Eq. (\ref{MaxBianchi}) are known as the Maxwell equations coming from the Bianchi identity.

  When a change can be made to the geometrical object, $A$ in this case, and the physical fields (what can be measured) do not vary, $\vec{E}$ and $\vec{B}$, the theory is said to posses a  {\bf gauge  invariance}.

  The electromagnetic theory is a gauge theory. Note that the electric and magnetic fields enters through $F$, not through $A$. Thus, if one changes $A\mapsto A+\df\, f$, the $F$ field do not changes,
  \begin{align}
    F\to F' = \df\, A' = \df\,\( A+\df\, f\) = \df\, A = F.
  \end{align}

  In order to conclude, it is worth to remark that the physical observables of the electromagnetic theory, $\vec{E}$ and $\vec{B}$, lie in a close form, $F$, but are defined up to a gauge transformation, i.e., $A$ is unique up to an exact form. Therefore, the physical states live in the second cohomological group of the Minkowski space, $H^2\(\R^{1,3}\)$.
\end{WEbox}


\section{Lie Groups and Lie Algebras}

\subsection{Lie Groups}

\begin{Def}[Group]
  A {\sc Group} $G$ is a set of elements, $\{g\}$, together with an operator, $\cdot: G\to G$, satisfying
  \begin{enumerate}
  \item Exists an unique identity element, $e$, s.t. $e\cdot g=g\cdot e =g$ for all $g\in G$.
  \item For every pair $g_1,g_2\in G$, the product $g_1\cdot g_2\equiv g_3$ belongs to $G$.
  \item Associativity: $g_1\cdot(g_2\cdot g_3)= (g_1\cdot g_2)\cdot g_3$, for all $g_1,g_2,g_3\in G$.
  \item Exists an inverse $g^{-1}\in G\; \forall g\in G$ s.t. $g\cdot g^{-1}=g^{-1}\cdot g=e$.
  \end{enumerate}
\end{Def}


