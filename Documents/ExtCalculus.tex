%%%%%%%%%%%%%%%%%% DIFFERENTIAL FORMS
\chapter{Exterior Calculus}

\section{Differential Forms}

\begin{Def}[Differential Forms]
  A \textsc{Differential Form} of order $p$, or $p$-form, is a totally antisymmetric tensor of type $\binom{0}{p}$.
\end{Def}

Since differential form are tensor, the basis could be the same than for tensors, i.e.,
\begin{align*}
  \set{\pa{\mu}},\set{\pa{\mu}\otimes\pa{\nu}},\set{\pa{\mu}\otimes\pa{\nu}\otimes\pa{\lambda}},\cdots,
\end{align*}
however, since forms are totally antisymmetric, it is more appealing to define an antisymmetric product of the basis. This new product is called ``wedge'' product \index{product!wedge}
($\we$), and is defined by
\begin{align}
  \de{x}^\mu\we\de{x}^\nu &= \de{x}^\mu\otimes\de{x}^\nu-\de{x}^\nu\otimes \de{x}^\mu\\
 &\vdots\\
  \de{x}^{\mu_1}\we\cdots\we\de{x}^{\mu_p}&= \sum_{\sigma\in S_p}(-1)^{\abs{\sigma}}\de{x}^{\sigma(\mu_1)}\otimes\cdots\otimes\de{x}^{\sigma(\mu_p)}
\end{align}

The vector space of differential forms of order $p$ on a manifold $M$ is denoted by $\Omega^p(M)$ or $\Lambda^p(M)$. Moreover, if $\dim(M)=m$, then
\begin{align}
  \dim\(\Lambda^p(M)\)=\binom{m}{p} =\frac{m!}{p!(m-p)!}.
\end{align}

\begin{infobox}
  In general $\Lambda^p(M)\subset \otimes^p T^*M$.
\end{infobox}

\subsection[Exterior Product]{Exterior Product (or wedge product)}

The exterior product \index{product!exterior} is a map 
\begin{align}
  \we:\Lambda^p(M) \times \Lambda^q(M) \to \Lambda^{p+q}(M)
\end{align}
defined by
\begin{align}
  (\omega\we\xi)\(V_1,\cdots,V_{p+q}\) = \frac{1}{p!q!}\sum_{\sigma\in S_p}(-1)^{\abs{\sigma}}\omega\(V_{\sigma(1)},\cdot,V_{\sigma(p)}\)\xi\(V_{\sigma(p+1)},\cdot,V_{\sigma(p+q)}\)
\end{align}

This exterior product endows an algebra structure to 
\begin{align}
  \Lambda^\bullet(M) \equiv \Lambda^0(M) \oplus \Lambda^1(M) \oplus \cdots \oplus \Lambda^m(M).
\end{align}
Clearly,
\begin{align}
  \dim\(\Lambda^\bullet(M)\) = 2^m.
\end{align}

\subsection{Exterior Derivative}

As stated previously, although the derivative transforms like a 1-form, the derivative of a tensor is not necessarily a tensor. However, it is possible to define a ``covariant'' derivative which maps tensors into tensors.

In the same way, one defines an exterior derivative\index{derivative!exterior}
which maps $p$-forms into $(p+1)$-forms,
\begin{align}
  \de{ :} \Lambda^p(M)\to \Lambda^{p+1}(M),
\end{align}
defined as follows. Let $\omega$ be a differential $p$-form on $M$, then
\begin{align}
  \de{\omega}\(X_1,\cdots,X_{p+1}\) &= \sum_{i=1}^p (-1)^{i+1}X_i\[\omega\(X_1,\cdots,\hat{X}_i,\cdots,X_{p+1}\)\]\\
  &\phantom{=} + \sum_{i<j} (-1)^{i+j}\omega\(\comm{X_i}{X_j},X_1,\cdots,\hat{X}_i,\cdots\hat{X}_i,\cdots,X_{p+1}\).\notag
\end{align}

\begin{Ebox}
  \begin{itemize}
  \item Show by explicit calculation that
    \begin{align}
      \de{\omega}_{(1)}(X,Y) = X[\omega(Y)]-Y[\omega(X)]-\omega(\comm{X}{Y}).
    \end{align}
  \item Show that if $\omega$ is a differential $p$-form with components
    \begin{align}
      \omega = \frac{1}{p!}\omega_{\mu_1\cdots\mu_p}\;\de{x}^{\mu_1}\we\cdots\we\de{x}^{\mu_p},
    \end{align}
    then its exterior derivative yields,
    \begin{align}
      \de{\omega} = \frac{1}{p!}\pa{\nu}\omega_{\mu_1\cdots\mu_p}\;\de{x}^\nu\we\de{x}^{\mu_1}\we\cdots\we\de{x}^{\mu_p}.
    \end{align}
  \end{itemize}
\end{Ebox}


It follows from the definition of the exterior derivative, that the double action of $\de{\,}$ vanishes, i.e., $\de{\,}^2=0$. This property is known as \emph{\idx{nilpotency}}.





\begin{WEbox}[frametitle={Differential Forms in $\R^3$},
  frametitlerule=true,
  frametitlealignment=\centering,
  frametitleaboveskip=10pt,]
  In three dimensions the maximum order of a differential form is 3. Therefore, the complete set of forms on $\R^3$ is given by
  \begin{align*}
    \Lambda^0(\R^3) &= \Fg(\R^3), \\
    \Lambda^1(\R^3) &= T^*\R^3 \simeq \R^3, \\
    \Lambda^2(\R^3) &= T^*\R^3 \we T^*\R^3 \simeq \R^3, \\
    \Lambda^3(\R^3) &= T^*\R^3 \we T^*\R^3 \we T^*\R^3 \simeq \Fg(\R^3).
  \end{align*}
  
  Let $f\in\Lambda^0(\R^3) $, then the exterior derivative of $f$ yields
  \begin{align}
    \de{f}= \frac{\partial f}{\partial x}\de{x} +\frac{\partial f}{\partial y}\de{y} + \frac{\partial f}{\partial z} \de{z},
  \end{align}
  which is the equivalent of the gradient operation.
  
  Let $\omega\in \Lambda^1(\R^3)$ with components $\omega = \omega_x \de{x} + \omega_y \de{y} + \omega_z \de{z}$, then its exterior derivative yields,
  \begin{align}
    \de{\omega} &= \(\partial_x\omega_y-\partial_y\omega_x\)\de{x}\we\de{y} +\(\partial_y\omega_z-\partial_z\omega_y\)\de{y}\we\de{z}\notag\\
    &\phantom{==}+\(\partial_z\omega_x-\partial_x\omega_z\)\de{z}\we\de{x}, 
  \end{align}
  which is the equivalent of the curl of $\omega$.

  Let $\omega\in \Lambda^2(\R^3)$ with components $\omega = \omega_{xy}\de{x}\we\de{y}+\omega_{yz}\de{y}\we\de{z}+\omega_{zx}\de{z}\we\de{x}$, then its exterior derivative yields,
  \begin{align}
    \de{\omega} = \(\partial_x\omega_{yz}+\partial_y\omega_{zx}+\partial_z\omega_{xy}\)\de{x}\we\de{y}\we \de{z},
  \end{align}
  which is the equivalent of the divergence of $\omega$.

  Finally, for $\omega\in\Lambda^3(\R^3)$, then $\de{\omega}=0$ due to the nilpotency of $\de{\,}$.


  Now, if one would like to compare with some differential properties on vector calculus, from the nilpotence of $\de{\,}$, it follows that
  \begin{align}
    \de{(\de{f})} =0 \quad &\Rightarrow \quad \vec{\nabla}\times \vec{\nabla}f =0,\\
    \de{\(\de{\omega}_{(1)}\)}=0 \quad&\Rightarrow\quad  \vec{\nabla}\cdot\( \vec{\nabla}\times \vec{\omega}\)=0.
  \end{align}
\end{WEbox}


\subsection{The Pullback}

A map $f:M\to N$ induces the \idx{pullback}
\begin{align}
  f^*:T^*_{f(p)}N\to T^*_pM,
\end{align}
which is extended naturally to type $\binom{0}{p}$-tensors (and to differential $p$-forms),
\begin{align}
  (f^*\omega)(V_1,\cdots,V_p) =\omega(f_*V_1,\cdots,f_*V_p).
\end{align}

\begin{Ebox}
  \begin{itemize}
  \item Show that
    \begin{align}
      f^*\(\de{\omega}\) =\de{\(f^*\omega\)}.
    \end{align}
    \item Show that
    \begin{align}
      f^*\(\omega\we\xi\) = (f^*\omega) \we (f^*\xi).
    \end{align}
  \end{itemize}
\end{Ebox}


\begin{WEbox}[frametitle={Cohomology},
  frametitlerule=true,
  frametitlealignment=\centering,
  frametitleaboveskip=10pt,]
  The exterior derivative operator induces a sequence, {\it a.k.a.} a complex,
  \begin{align}
    0\stackrel{~i~}{\hookrightarrow }\Lambda^0(M) \xrightarrow{~\de{\,}~}\Lambda^1(M) \xrightarrow{~\de{\,}~}\cdots\xrightarrow{~\de{\,}~} \Lambda^{m-1}(M) \xrightarrow{~\de{\,}~}\Lambda^m(M) \xrightarrow{~\de{\,}~} 0,
  \end{align}
  called the {\bf de Rham complex}\index{complex!de Rham}.

  Defining,
  \begin{itemize}
  \item A form, $\omega$, is said to be closed if $\de{\omega} =0$.
  \item A form, $\omega$, is said to be exact if $\omega =\de{\xi}$.
  \end{itemize}

  Due to the nilpotency of $\de{,}$ it follows that,
  {\small
  \begin{align*}
    \Im\(\de{:}C^\infty\(\Lambda^{p-1}(M)\)\to C^\infty\(\Lambda^{p}(M)\)\)\subset\Ker\(\de{:}C^\infty\(\Lambda^{p}(M)\)\to C^\infty\(\Lambda^{p+1}(M)\) \).
  \end{align*}
  }

  Therefore, the {\bf $\mathbf{p}$-th de Rham Cohomology group}\index{cohomology!de Rham}, $H^p(M)$, is defined by
  \begin{align*}
    H^p(M)\equiv \frac{\Ker\(\de{:}C^\infty\(\Lambda^{p}(M)\)\to C^\infty\(\Lambda^{p+1}(M)\) \)}{\Im\(\de{:}C^\infty\(\Lambda^{p-1}(M)\)\to C^\infty\(\Lambda^{p}(M)\)\)}.
  \end{align*}
\end{WEbox}


\subsection{Interior Product and Lie Derivative}

The interior product\index{product!interior} is  a map
\begin{align}
  i:TM\times \Lambda^p(M)\to \Lambda^{p-1}(M),
\end{align}
or usually denoted
\begin{align}
  i_X:\Lambda^p(M)\to \Lambda^{p-1}(M),
\end{align}
defined by
\begin{align}
  i_X\omega(V_1,\cdots,V_{p-1})=\omega(X,V_1,\cdots,V_{p-1}).
\end{align}
In components it can be written as
\begin{equation}
  \begin{split}
    i_X\omega &= \frac{1}{(p-1)!} X^\nu \omega_{\nu \mu_2 \cdots\mu_{p}}\;\de{x}^{\mu_2}\we\cdots\we\de{x}^{\mu_p}\\
    &= \frac{1}{p!} \sum_{s=1}^p (-1)^{s-1} X^{\mu_s} \omega_{\mu_1 \cdots\mu_s \cdots \mu_{p}}\;\de{x}^{\mu_2}\we\cdots\we\widehat{\de{x}}^{\mu_s}\we\cdots\we\de{x}^{\mu_p}.
  \end{split}
  \label{ext-prod}
\end{equation}

\begin{Ebox}
  Prove the equivalence between the two expressions of the interior product in Eq.~\eqref{ext-prod}.
\end{Ebox}

Using the exterior derivative and the interior product, one can construct the operator \mbox{$\de{i}_x+i_X \de{.}$} This operator can be applied to 1-forms,
\begin{align}
  \(\de{i}_X + i_X \de{\,}\)\omega &=\de{(X^\mu\omega_\mu)} + i_X\(\pa{\mu}\omega_\nu \de{x}^\mu\we\de{x}^\nu\)\notag\\
  &=\(\pa{\nu}X^\mu\)\omega_\mu\de{x}^\nu +X^\mu \(\pa{\nu}\omega_\mu\) \de{x}^\nu + X^\mu \(\pa{\mu}\omega_\nu - \pa{\nu}\omega_\mu\) \de{x}^\nu \notag\\
  &=\[\omega(\pa{\nu}X^\mu)+X^\mu(\pa{\mu}\omega_\nu)\]\de{x}^\nu,
\end{align}
which is nothing but the Lie derivative of a differential 1-form,
\begin{align}
  \Li_X\omega = \(\de{i}_x + i_X \de{\,}\)\omega.
\end{align}
The above expression is general for differential forms of any order.

The interior product and the Lie derivative satisfy
\begin{align}
  i_X \(\omega \we \eta\) &= \(i_X\omega\) \we \eta + (-1)^{\abs{\omega}} \omega \we \(i_X\eta\)
  \label{iwedge} \\
  (i_X)^2 &= 0
  \label{isquare} \\
  \Li_X i_X \omega &= i_X \Li_X \omega.
  \label{iLie}
\end{align}

\begin{Ebox}
  Prove the equalities in Eqs.~\eqref{iwedge}, \eqref{isquare} and \eqref{iLie}.
\end{Ebox}


\begin{WEbox}[frametitle={Geometry of Classical Mechanics\index{machanics!classical}},
  frametitlerule=true,
  frametitlealignment=\centering,
  frametitleaboveskip=10pt,]
  Let $M$ be a $m$-dimensional manifold, and $\Phi^m\equiv T^*M$ be its cotangent bundle (or \emph{phase space}). A  \emph{\idx{Hamiltonian}} on $M$ is a map $\Ha:\Phi^m\to \R$.

  Let $\set{q^\mu,p_\mu}$, with $\mu = 1,\cdots, m$, be  the coordinates on $\Phi^m$. One might define a \idx{symplectic form} on $\Phi^m$ by
  \begin{align*}
    \Lambda^2\(\Phi^m\) \ni \omega = \de{p}_\mu \we \de{q}^\mu,
  \end{align*}
  and a 1-form $\theta=q^\mu \de{p}_\mu\in \Lambda^1\(\Phi^m\)$, s.t. $\omega=-\de{\theta}$.
  
  Given a function $f:\Phi^m\to \R$, one might define the \textsc{Hamiltonian Vector Field}\index{Hamiltonian!vector field}
  \begin{align*}
    X_f = \pder{f}{p_\mu}\pder{}{q^\mu} -\pder{f}{q^\mu}\pder{}{p_\mu}.
  \end{align*}
  Then,
  \begin{align*}
    i_{X_f}\omega =-\pder{f}{q^\mu}\de{q}^\mu -\pder{f}{p_\mu}\de{p}_\mu = -\de{f}.
  \end{align*}

  The Hamilton equations,
  \begin{align*}
    \dot{q}^\mu =\pder{\Ha}{p_\mu},\quad \dot{p}_\mu =-\pder{\Ha}{q^\mu},
  \end{align*}
  can be written as
  \begin{align*}
    i_{X_\Ha}\omega = -\pder{\Ha}{q^\mu}\de{q}^\mu -\pder{\Ha}{p_\mu}\de{p}_\mu = -\de{\Ha},
  \end{align*}
  where the identity
  \begin{align*}
    X_\Ha &=\pder{\Ha}{p_\mu}\pder{}{q^\mu} -\pder{\Ha}{q^\mu}\pder{}{p_\mu}\\
    &= \dot{q}^\mu\pder{}{q^\mu} +\dot{p}_\mu\pder{}{p_\mu}\\
    &= \der{}{t},
  \end{align*}
  has been used.

  Also, the action of two interior products on $\omega$ yields the \idx{Poisson bracket},
  \begin{align*}
    i_{X_f}\(i_{X_g}\omega\) &=-i_{X_f}(\de{g}) = -i_{X_f}\(\pder{g}{q^\mu}\de{q}^\mu +\pder{g}{p_\mu}\de{p}_\mu\)\\
    &= -\pder{f}{p_\mu}\pder{g}{q^\mu}+\pder{f}{q^\mu}\pder{g}{p_\mu}\\
    &= \acomm{f}{g}_{P.B.}.
  \end{align*}
\end{WEbox}

\begin{WEbox}[%
    frametitle={Gauge Theory (Abelian)},
    frametitlerule=true,
    frametitlealignment=\centering,
    frametitleaboveskip=10pt,]
  Let $M$ be a four-dimensional, flat, Lorentzian spacetime endowed with a Minkowski metric\index{gauge!theory (Abelian)}, $\eta=\diag(-1,1,1,1)$, and $A_\mu = (-\phi,\vec{A})$ be a four-vector on $T^*M$. One can construct the 1-form 
  \begin{align}
    A = -\phi \de{t} + A_i \de{x}^i.
  \end{align}

  Define the 2-form, $F$ as the exterior derivative of $A$,
  \begin{align}
    F &= \de{A}\\
      &= - \partial_i\phi\;\de{x}^i\we\de{t} + \partial_t A_i\;\de{t}\we\de{x}^i + \partial_i A_j\;\de{x}^i\we\de{x}^j\\
      &= \(\partial_i\phi+\partial_t A_i\)\;\de{t}\we\de{x}^i+ \partial_i A_j\;\de{x}^i\we\de{x}^j\\
      &= -\de{t} \we \mathbf{E}_{(1)} +\mathbf{B}_{(2)},
  \end{align}
  where the differential forms $\mathbf{E}$ and $\mathbf{B}$ are the associated to the electric and magnetic field.
  
  Now, the nilpotency of $\de{\,}$ implies that,
  \begin{align}
    \de{F} = \de{(\de{A})} = 0,
  \end{align}
  while on the other hand, 
  \begin{align}
    \de{F} = \de{t}\we \de{_s}\mathbf{E}_{(1)} +\de{_t}\mathbf{B}_{(2)}+\de{_s}\mathbf{B}_{(2)},\label{MaxdF}
  \end{align}
  where $\de{_t}$ and $\de{_s}$ are the exterior derivative on the time and space respectively.
  
  Since the basis elements $\de{t}$ and $\de{x}^i$ are linearly independent, it follows that components containing projections along the time are independent of those lying only on the space. Thus, Eq. \eqref{MaxdF}, decompose into two independent equations,
  \begin{align}
    \partial_t \vec{B} + \vec{\nabla} \cdot \vec{E} &= 0, \notag\\
    \vec{\nabla} \cdot \vec{B}&=0,
    \label{MaxBianchi}
  \end{align}
  where the relation between the exterior derivative on forms and the vector calculus have been used. The Eq. \eqref{MaxBianchi} are known as the Maxwell equations coming from the Bianchi identity.

  When a change can be made to the geometrical object, $A$ in this case, and the physical fields (what can be measured) do not vary, $\vec{E}$ and $\vec{B}$, the theory is said to posses a  \emph{gauge  invariance}\index{gauge!invariance}.

  The electromagnetic theory is a gauge theory. Note that the electric and magnetic fields enters through $F$, not through $A$. Thus, if one changes $A\mapsto A + \de{f}$, the $F$ field do not changes,
  \begin{align}
    F \to F' = \de{A'} = \de{\( A + \de{f}\)} = \de{A} = F.
  \end{align}

  In order to conclude, it is worth to remark that the physical observables of the electromagnetic theory, $\vec{E}$ and $\vec{B}$, lie in a close form, $F$, but are defined up to a gauge transformation, i.e., $A$ is unique up to an exact form. %Therefore, the physical states live in the second cohomological group of the Minkowski space, $H^2\(\R^{3,1}\)$.
\end{WEbox}


\section{Lie Groups and Lie Algebras}

\subsection{Lie Groups}

\begin{Def}[Group]
  A  \textsc{Group}\index{group} $G$ is a set of elements, $\{g\}$, together with an operator, $\cdot: G \times G \to G$, satisfying
  \begin{itemize}
  \item \emph{Identity} Exists an unique identity element, $e$, s.t. $e \cdot g = g \cdot e =g$ for all $g \in G$.
  \item \emph{Closure} For every pair $g_1, g_2\in G$, the product $g_1\cdot g_2 = g_3$ belongs to $G$.
  \item \emph{Associativity} $g_1 \cdot (g_2 \cdot g_3) = (g_1 \cdot g_2) \cdot g_3$, for all $g_1$, $g_2$, $g_3$ in $G$.
  \item \emph{Inverse} Exists a unique inverse $g^{-1} \in G\; \forall g \in G$ s.t. $g \cdot g^{-1} = g^{-1} \cdot g = e$.
  \end{itemize}
\end{Def}

\begin{Def}[Lie Group]
  A \textsc{Lie Group}\index{Lie group}, $G$, is a differential manifold endowed with a group structure
  \begin{equation}
    \begin{split}
      \cdot: G \times G &\to G \\
      (g_1,g_2) &\mapsto g_1 \cdot g_2
    \end{split}
    \quad
    \text{and}
    \quad
    \begin{split}
      (\ )^{-1}: G &\to G \\
      g_1 &\mapsto g^{-1}
    \end{split}
  \end{equation}
\end{Def}

\begin{Exa}
  \mbox{}
  \begin{itemize}
  \item $(\R^>,\cdot)$
  \item $(\R^n,+)$
  \item $GL(n;\K)$
  \end{itemize}
\end{Exa}

\begin{Thm}\label{thm:LieSubgroup}
  Every closed subgroup $H$ of a Lie group $G$ is a Lie group.
\end{Thm}

Accordingly to the Theorem~\ref{thm:LieSubgroup}, the subgroups of $GL(n,\K)$ are Lie groups, for example:
\begin{equation*}
  GL(n,\K) \supset SL(n,\K) \supset SU(n,\K).
\end{equation*}

If $G$ is a Lie group and $H \subset G$, one might define an \emph{equivalence relation}\index{relation!equivalence}, denoted by $\sim$, such that $g \sim g'$ if exists an element $h \in H$ satisfying that $g' = g \cdot h$.

The set of all ``equivalent'' elements is denoted by $[g]$ and is called an \emph{equivalence class}\index{class!equivalence}.

The coset group, $G/H$, is also a manifold called \emph{coset manifold}\index{manifold!coset}. Additionally, if $H$ is a normal subgroup of $G$, then the coset manifold posses a Lie group structure.


\subsection{Action of Lie groups on Manifolds}

The action of a group\index{Lie group!action} on a manifold is a generalisation of the concept of flow.
\begin{Def}
  Let $G$ be a Lie group and $M$ be a manifold. The action of $G$ on $M$ is a differential map \mbox{$\sigma: G \times M \to M$} satisfying
  \begin{align}
    \sigma(e, p) &= p, \quad \forall p \in M ,\\
    \sigma(g_1, \sigma(g_2, p )) &= \sigma(g_1 \cdot g_2, p),
  \end{align}
  where $e$ is the identity element of $G$, and the dot ($\cdot$) is the group operation of $G$.
\end{Def}

The action $\sigma$ is said to be 
\begin{itemize}
\item Transitive\index{Lie group!action!transitive}: If for all pair of points on $M$, say $p_1$ and $p_2$, there exists an element $g \in G$ such that $\sigma(g, p_1) = p_2$.
\item Free\index{Lie group!action!free}: If for an element $g \in G$  there exists an $p \in M$ with $\sigma(g, p) = p$ (that is, if $g$ has at least one \idx{fixed point}), then $g$ is the identity.
\item Effective\index{Lie group!action!effective}: If for any $g \neq e \in G$ there exists an $p \in M$ such that $\sigma(g, p) \neq p$.
\end{itemize}

Using the concept of action, one can define the \emph{\idx{orbit}} of $p \in M$ under the action of the group $G$, as the subset  $Gp \subset M$ such that
\begin{equation}
  Gp = \set{\sigma(g, p) \mid g \in G}.
\end{equation}
Also, the \idx{isotropy group} of $p \in M$ is a subgroup $H \subset G$, defined by
\begin{equation}
  H(p) = \set{g \in G \mid \sigma(g, p) = p}.
\end{equation}
The isotropy group is also known as \emph{little group} or \emph{stabiliser} of $p$.

It is customary, for the sake of notation simplification, to denote the action $\sigma(g, p)$ by $g p$.


\subsection{Lie Algebra}

One might define the action of an element $g \in G$ on a manifold in two ways,
\begin{equation}
  \begin{split}
    L_g p &= p g, \\
    R_g p &= g p,
  \end{split}
\end{equation}
called the left and right action of $G$ on $M$ respectively.

These maps induce actions on the tangent space of $G$ at $g$, $T_g G$, defined by
\begin{equation}
  \begin{split}
    L_{a*} : T_g G &\to T_{ga} G, \\
    R_{a*} : T_g G &\to T_{ag} G.
  \end{split}
\end{equation}

\begin{Def}
  A vector field $X$ on $G$, $X \in TG$, is said to be a \emph{left invariant} vector field\index{vector field!left invariant} if
  \begin{equation}
    L_{a*} X \bigr|_{g} = X \bigr|_{ag}.
  \end{equation}
\end{Def}

\begin{WEbox}[%
    frametitle={Left Invariant Vector Field},
    frametitlerule=true,
    frametitlealignment=\centering,
    frametitleaboveskip=10pt,]
  Let $M$ be a manifold and $p$ a point on $M$ with coordinates $x$. Let $X = \pa{x}$ be a vector field on $M$, and the group of translations acts on $M$ through 
  \begin{equation}
    L_a x = x + a.
  \end{equation}
  The induced action on the vector $X$ is
  \begin{equation}
    L_{a*} X \bigr|_x = \pder{}{x}(x + a) \pder{}{(x + a)} = \pder{}{(x + a)} = X \bigr|_{x+a},
  \end{equation}
  Therefore, $X$ is a left invariant vector field under the action of translations on $M$.
\end{WEbox}

\begin{infobox}
  The action of the Lie group $G$ on a point $g$ in the manifold $G$ is trivially defined by the group operation, and from the axioms of group, the result is a new point on $G$.

  Therefore, left invariant vector fields are vectors which are globally defined on the tangent bundle of the group manifold, $TG$. 
\end{infobox}

According to Eq.~\eqref{diff-comm}, if $X$ and $Y$ are left invariant vector fields, it follows that
\begin{equation}
  L_{a*} \comm{X}{Y} \bigr|_{g} = \comm{X}{Y} \bigr|_{ag},
  \label{left-comm}
\end{equation}
\emph{i.e.}, the Lie bracket of left invariant vector fields is a left invariant vector field.



Since the set left invariant vector fields, $\mathfrak{g}$, are well defined over $TG$, they serve to define a basis on $TG$. Moreover, it is enough to define them at a single point of the manifold (because they are invariant), say $T_e G$.

One then define an isomorphism between $T_e G$ and $\mathfrak{g}$, and it yields
\begin{equation*}
  \dim(\mathfrak{g}) = \dim(G).
\end{equation*}
Additionally, from Eq.~\eqref{left-comm}, it follows that $\mathfrak{g}$ is closed under the action of the Lie bracket.

\begin{Def}
  [Lie Algebra]
  The set of left invariant vector fields, $\mathfrak{g}$, with the Lie bracket $\comm{\ }{\ }: \mathfrak{g} \times \mathfrak{g} \to \mathfrak{g}$ form an algebra called \emph{\idx{Lie algebra}} of the Lie group $G$.
\end{Def}

Consider a set of left invariant vector fields $\mathfrak{g} = \{X_i \bigr|_g\}$. Since the Lie bracket $\comm{X_a}{X_b} \bigr|_g$  is another left invariant vector fields, it lives in $\mathfrak{g}$. Therefore, it should be a linear combination of the basis $\mathfrak{g}$, \emph{i.e.},
\begin{equation*}
  \comm{X_a}{X_b} \bigr|_g = C_{ab}{}^c(g)\, X_c \bigr|_g.
\end{equation*}
Considering that the left invariant vector fields are well defined globally, the above equation should be independent of the point $g$. Thus,
\begin{equation*}
  C_{ab}{}^c(g) = C_{ab}{}^c,
\end{equation*}
are in fact constants, know are \emph{\idx{structure constants}} of the Lie group $G$.

The Lie algebra is then characterised by the relations,
\begin{equation}
  \comm{X_a}{X_b} = C_{ab}{}^c\, X_c .
  \label{Lie-alg}
\end{equation}

\begin{Ebox}
  Show that the structure constants satisfy 
  \begin{itemize}
  \item $C_{ab}{}^c = - C_{ba}{}^c$.
  \item $C_{ab}{}^d C_{dc}{}^f + C_{bc}{}^d C_{da}{}^f + C_{ca}{}^d C_{db}{}^f = 0$.
  \end{itemize}
\end{Ebox}


\subsection{Relation between 1-parameter subgroup and flows}

A curve $\phi: \R \to G$ is called a 1-parameter subgroup of $G$ if
\begin{equation*}
  \phi(t) \cdot \phi(s) = \phi(t+s).
\end{equation*}
It can be shown, that for a given 1-parameter subgroup $\phi$ of $G$, there exists a vector field $X \in TG$ such that 
\begin{equation*}
  \der{}{t} \phi(t) = X[\phi(t)].
\end{equation*}
Moreover, $X$ is a left-invariant vector field. It follows from Eq.~\eqref{sol-flow} that
\begin{equation}
  \phi_X(t) = \exp(t X).
\end{equation}

\section{Frames and Structure Equation}

In a few words, the structure equations of Maurer--Cartan are the dual equivalent to the Lie algebra relation [see Eq.~\eqref{Lie-alg}]. However, a detour will be taken to introduce several concepts.

\subsection{Matrix Valued Differential Forms}

A \emph{matrix valued differential form} is a  differential form which comes with a matrix attached to it. This can be thought also as a matrix whose entries are differential forms on $M$. Therefore, a matrix value differential $p$-form, $\Omega$, is a geometrical object living in \mbox{$\Lambda^p(M) \otimes \Mat(n)$,} with $\Mat(n)$ the space of $n$ by $n$ matrices.

Let $R$ and $S$ be matrix valued $p$-forms, operations on the exterior algebra and on matrices apply as follows
\begin{align}
  R &= \(R_{ij}\), \\
  \de{R} &= \( \de{R}_{ij} \), \\
  R \we S &= \( \sum_j R_{ij} \we S_{jk} \),
  \label{mvdf-w}
\end{align}
with $R_{ij}$ and $S_{ij}$ in $\Lambda^p(M)$.

Note that in Eq.~\eqref{mvdf-w}, there is a matrix product, and in a general vector space $\Mat(n)$ the multiplicative inverse is not defined. However, in the restriction $GL(n) \subset \Mat(n)$, it is possible to define the inverse, $R^{-1}$,  of a matrix values differential $p$-form $R$, such that 
\begin{equation}
  \begin{split}
    R &\in \Lambda^{p}(M) \otimes GL(n) \\
    R^{-1} &\in \Lambda^{p}(M) \otimes GL(n) \\
    \{ R \we R^{-1}, R^{-1} \we R \} &\in \Lambda^{2p}(M) \otimes \mathds{1}_n.
  \end{split}
\end{equation}

\begin{Ebox}
  Calculate $\de{A^{-1}}$, where $A^{-1}$ is the matrix valued function on $GL(n)$ assigning to each invertible matrix its inverse.
\end{Ebox}

\subsection{A Primer on Maurer--Cartan Form}

Let $G = GL(n)$  be the group of invertible, $n$ by $n$ matrices, and $A$ be a matrix valued $0$-form on $\Lambda^0(M) \otimes G$. The construction 
\begin{equation}
  \omega = A^{-1} \de{A},
  \label{MC-form}
\end{equation}
is a matrix valued $1$-form, \emph{i.e.}, $\omega \in \Lambda^1(M) \otimes G$.

\begin{Ebox}
  Consider $\omega = A^{-1} \de{A}$. For a fixed $g \in G$, show that:
  \begin{itemize}
  \item Under left action, $\omega$ is an invariant $1$-form, \emph{i.e.},
    \begin{equation}
      L_g^* \omega = \omega.
    \end{equation}
  \item Under right action, $\omega$ transforms in the adjoint representation, \emph{i.e.},
    \begin{equation}
      R_g^* \omega = g\omega g^{-1}.
    \end{equation}
  \item The following relation is satisfied,
    \begin{equation}
      \de{\omega} + \omega \we \omega =0.
      \label{MC-eq-pre}
    \end{equation}
  \end{itemize}
\end{Ebox}

The matrix valued $1$-form in Eq.~\eqref{MC-form} is called the \emph{Maurer--Cartan form}\index{Maurer--Cartan!form}, while the relation in Eq.~\eqref{MC-eq-pre} is called the \emph{Maurer--Cartan equation} of structure\index{Maurer--Cartan!equation}.

\subsubsection{Restrictions to a Subgroup of $GL(n)$}

If $H$ is a Lie subgroup of $G$, then $H$ is a (sub)manifold. 

\subsection{Frames}

\subsubsection{Euclidean Frames}

\begin{Ebox}
  Using
  \begin{enumerate}
  \item $\de{e}^a = \Omega^a{}_b e^b$,
  \item $\de{u} = e^a \theta_a$,
  \end{enumerate}
  where the notation was defined on class, derive the Structure Equations for the Euclidean Geometry,
  \begin{align}
    \de{\theta}^a + \Omega^a{}_b \we \theta^b &= 0,\\
    \de{\Omega}^a{}_c + \Omega^a{}_b \we \Omega^b{}_c &= 0.
  \end{align}
\end{Ebox}
